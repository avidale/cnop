
\documentclass[letterpaper,fleqn,11pt]{article}
%%%%%%%%%%%%%%%%%%%%%%%%%%%%%%%%%%%%%%%%%%%%%%%%%%%%%%%%%%%%%%%%%%%%%%%%%%%%%%%%%%%%%%%%%%%%%%%%%%%%%%%%%%%%%%%%%%%%%%%%%%%%%%%%%%%%%%%%%%%%%%%%%%%%%%%%%%%%%%%%%%%%%%%%%%%%%%%%%%%%%%%%%%%%%%%%%%%%%%%%%%%%%%%%%%%%%%%%%%%%%%%%%%%%%%%%%%%%%%%%%%%%%%%%%%%%
\usepackage{booktabs}
\usepackage{multirow}
\usepackage{geometry}
\usepackage[singlespacing]{setspace}
\usepackage{amsfonts}
\usepackage{inputenc}
\usepackage{graphicx}
\usepackage{amsmath}
\usepackage{accents}
\usepackage{eurosym}
\usepackage{amssymb}
\usepackage{rotating}
\usepackage{sectsty}
\usepackage{endnotes}
\usepackage{chbibref}
\usepackage{float}
\usepackage{nopageno}
\usepackage[labelsep=period]{caption}
\usepackage{scalefnt}

\setcounter{MaxMatrixCols}{10}
%TCIDATA{OutputFilter=LATEX.DLL}
%TCIDATA{Version=5.50.0.2960}
%TCIDATA{<META NAME="SaveForMode" CONTENT="2">}
%TCIDATA{BibliographyScheme=Manual}
%TCIDATA{Created=Wednesday, April 07, 2010 09:52:31}
%TCIDATA{LastRevised=Monday, August 21, 2017 22:00:12}
%TCIDATA{<META NAME="GraphicsSave" CONTENT="32">}
%TCIDATA{<META NAME="DocumentShell" CONTENT="Standard LaTeX\Blank - Standard LaTeX Article">}
%TCIDATA{Language=American English}
%TCIDATA{CSTFile=40 LaTeX article.cst}

\newtheorem{theorem}{Theorem}
\newtheorem{acknowledgement}[theorem]{Acknowledgement}
\newtheorem{algorithm}[theorem]{Algorithm}
\newtheorem{axiom}[theorem]{Axiom}
\newtheorem{case}[theorem]{Case}
\newtheorem{claim}[theorem]{Claim}
\newtheorem{conclusion}[theorem]{Conclusion}
\newtheorem{condition}[theorem]{Condition}
\newtheorem{conjecture}[theorem]{Conjecture}
\newtheorem{corollary}[theorem]{Corollary}
\newtheorem{criterion}[theorem]{Criterion}
\newtheorem{definition}[theorem]{Definition}
\newtheorem{example}[theorem]{Example}
\newtheorem{exercise}[theorem]{Exercise}
\newtheorem{lemma}[theorem]{Lemma}
\newtheorem{notation}[theorem]{Notation}
\newtheorem{problem}[theorem]{Problem}
\newtheorem{proposition}[theorem]{Proposition}
\newtheorem{remark}[theorem]{Remark}
\newtheorem{solution}[theorem]{Solution}
\newtheorem{summary}[theorem]{Summary}
\newenvironment{proof}[1][Proof]{\noindent\textbf{#1.} }{\ \rule{0.5em}{0.5em}}
\geometry{left=1.5 in,right=1 in,top=1 in,bottom=1.1 in}
% Macros for Scientific Word 2.5 documents saved with the LaTeX filter.
%Copyright (C) 1994-95 TCI Software Research, Inc.
\typeout{TCILATEX Macros for Scientific Word 2.5 <22 Dec 95>.}
\typeout{NOTICE:  This macro file is NOT proprietary and may be 
freely copied and distributed.}
%
\makeatletter
%
%%%%%%%%%%%%%%%%%%%%%%
% macros for time
\newcount\@hour\newcount\@minute\chardef\@x10\chardef\@xv60
\def\tcitime{
\def\@time{%
  \@minute\time\@hour\@minute\divide\@hour\@xv
  \ifnum\@hour<\@x 0\fi\the\@hour:%
  \multiply\@hour\@xv\advance\@minute-\@hour
  \ifnum\@minute<\@x 0\fi\the\@minute
  }}%

%%%%%%%%%%%%%%%%%%%%%%
% macro for hyperref
\@ifundefined{hyperref}{\def\hyperref#1#2#3#4{#2\ref{#4}#3}}{}

% macro for external program call
\@ifundefined{qExtProgCall}{\def\qExtProgCall#1#2#3#4#5#6{\relax}}{}
%%%%%%%%%%%%%%%%%%%%%%
%
% macros for graphics
%
\def\FILENAME#1{#1}%
%
\def\QCTOpt[#1]#2{%
  \def\QCTOptB{#1}
  \def\QCTOptA{#2}
}
\def\QCTNOpt#1{%
  \def\QCTOptA{#1}
  \let\QCTOptB\empty
}
\def\Qct{%
  \@ifnextchar[{%
    \QCTOpt}{\QCTNOpt}
}
\def\QCBOpt[#1]#2{%
  \def\QCBOptB{#1}
  \def\QCBOptA{#2}
}
\def\QCBNOpt#1{%
  \def\QCBOptA{#1}
  \let\QCBOptB\empty
}
\def\Qcb{%
  \@ifnextchar[{%
    \QCBOpt}{\QCBNOpt}
}
\def\PrepCapArgs{%
  \ifx\QCBOptA\empty
    \ifx\QCTOptA\empty
      {}%
    \else
      \ifx\QCTOptB\empty
        {\QCTOptA}%
      \else
        [\QCTOptB]{\QCTOptA}%
      \fi
    \fi
  \else
    \ifx\QCBOptA\empty
      {}%
    \else
      \ifx\QCBOptB\empty
        {\QCBOptA}%
      \else
        [\QCBOptB]{\QCBOptA}%
      \fi
    \fi
  \fi
}
\newcount\GRAPHICSTYPE
%\GRAPHICSTYPE 0 is for TurboTeX
%\GRAPHICSTYPE 1 is for DVIWindo (PostScript)
%%%(removed)%\GRAPHICSTYPE 2 is for psfig (PostScript)
\GRAPHICSTYPE=\z@
\def\GRAPHICSPS#1{%
 \ifcase\GRAPHICSTYPE%\GRAPHICSTYPE=0
   \special{ps: #1}%
 \or%\GRAPHICSTYPE=1
   \special{language "PS", include "#1"}%
%%%\or%\GRAPHICSTYPE=2
%%%  #1%
 \fi
}%
%
\def\GRAPHICSHP#1{\special{include #1}}%
%
% \graffile{ body }                                  %#1
%          { contentswidth (scalar)  }               %#2
%          { contentsheight (scalar) }               %#3
%          { vertical shift when in-line (scalar) }  %#4
\def\graffile#1#2#3#4{%
%%% \ifnum\GRAPHICSTYPE=\tw@
%%%  %Following if using psfig
%%%  \@ifundefined{psfig}{\input psfig.tex}{}%
%%%  \psfig{file=#1, height=#3, width=#2}%
%%% \else
  %Following for all others
  % JCS - added BOXTHEFRAME, see below
    \leavevmode
    \raise -#4 \BOXTHEFRAME{%
        \hbox to #2{\raise #3\hbox to #2{\null #1\hfil}}}%
}%
%
% A box for drafts
\def\draftbox#1#2#3#4{%
 \leavevmode\raise -#4 \hbox{%
  \frame{\rlap{\protect\tiny #1}\hbox to #2%
   {\vrule height#3 width\z@ depth\z@\hfil}%
  }%
 }%
}%
%
\newcount\draft
\draft=\z@
\let\nographics=\draft
\newif\ifwasdraft
\wasdraftfalse

%  \GRAPHIC{ body }                                  %#1
%          { draft name }                            %#2
%          { contentswidth (scalar)  }               %#3
%          { contentsheight (scalar) }               %#4
%          { vertical shift when in-line (scalar) }  %#5
\def\GRAPHIC#1#2#3#4#5{%
 \ifnum\draft=\@ne\draftbox{#2}{#3}{#4}{#5}%
  \else\graffile{#1}{#3}{#4}{#5}%
  \fi
 }%
%
\def\addtoLaTeXparams#1{%
    \edef\LaTeXparams{\LaTeXparams #1}}%
%
% JCS -  added a switch BoxFrame that can 
% be set by including X in the frame params.
% If set a box is drawn around the frame.

\newif\ifBoxFrame \BoxFramefalse
\newif\ifOverFrame \OverFramefalse
\newif\ifUnderFrame \UnderFramefalse

\def\BOXTHEFRAME#1{%
   \hbox{%
      \ifBoxFrame
         \frame{#1}%
      \else
         {#1}%
      \fi
   }%
}


\def\doFRAMEparams#1{\BoxFramefalse\OverFramefalse\UnderFramefalse\readFRAMEparams#1\end}%
\def\readFRAMEparams#1{%
 \ifx#1\end%
  \let\next=\relax
  \else
  \ifx#1i\dispkind=\z@\fi
  \ifx#1d\dispkind=\@ne\fi
  \ifx#1f\dispkind=\tw@\fi
  \ifx#1t\addtoLaTeXparams{t}\fi
  \ifx#1b\addtoLaTeXparams{b}\fi
  \ifx#1p\addtoLaTeXparams{p}\fi
  \ifx#1h\addtoLaTeXparams{h}\fi
  \ifx#1X\BoxFrametrue\fi
  \ifx#1O\OverFrametrue\fi
  \ifx#1U\UnderFrametrue\fi
  \ifx#1w
    \ifnum\draft=1\wasdrafttrue\else\wasdraftfalse\fi
    \draft=\@ne
  \fi
  \let\next=\readFRAMEparams
  \fi
 \next
 }%
%
%Macro for In-line graphics object
%   \IFRAME{ contentswidth (scalar)  }               %#1
%          { contentsheight (scalar) }               %#2
%          { vertical shift when in-line (scalar) }  %#3
%          { draft name }                            %#4
%          { body }                                  %#5
%          { caption}                                %#6


\def\IFRAME#1#2#3#4#5#6{%
      \bgroup
      \let\QCTOptA\empty
      \let\QCTOptB\empty
      \let\QCBOptA\empty
      \let\QCBOptB\empty
      #6%
      \parindent=0pt%
      \leftskip=0pt
      \rightskip=0pt
      \setbox0 = \hbox{\QCBOptA}%
      \@tempdima = #1\relax
      \ifOverFrame
          % Do this later
          \typeout{This is not implemented yet}%
          \show\HELP
      \else
         \ifdim\wd0>\@tempdima
            \advance\@tempdima by \@tempdima
            \ifdim\wd0 >\@tempdima
               \textwidth=\@tempdima
               \setbox1 =\vbox{%
                  \noindent\hbox to \@tempdima{\hfill\GRAPHIC{#5}{#4}{#1}{#2}{#3}\hfill}\\%
                  \noindent\hbox to \@tempdima{\parbox[b]{\@tempdima}{\QCBOptA}}%
               }%
               \wd1=\@tempdima
            \else
               \textwidth=\wd0
               \setbox1 =\vbox{%
                 \noindent\hbox to \wd0{\hfill\GRAPHIC{#5}{#4}{#1}{#2}{#3}\hfill}\\%
                 \noindent\hbox{\QCBOptA}%
               }%
               \wd1=\wd0
            \fi
         \else
            %\show\BBB
            \ifdim\wd0>0pt
              \hsize=\@tempdima
              \setbox1 =\vbox{%
                \unskip\GRAPHIC{#5}{#4}{#1}{#2}{0pt}%
                \break
                \unskip\hbox to \@tempdima{\hfill \QCBOptA\hfill}%
              }%
              \wd1=\@tempdima
           \else
              \hsize=\@tempdima
              \setbox1 =\vbox{%
                \unskip\GRAPHIC{#5}{#4}{#1}{#2}{0pt}%
              }%
              \wd1=\@tempdima
           \fi
         \fi
         \@tempdimb=\ht1
         \advance\@tempdimb by \dp1
         \advance\@tempdimb by -#2%
         \advance\@tempdimb by #3%
         \leavevmode
         \raise -\@tempdimb \hbox{\box1}%
      \fi
      \egroup%
}%
%
%Macro for Display graphics object
%   \DFRAME{ contentswidth (scalar)  }               %#1
%          { contentsheight (scalar) }               %#2
%          { draft label }                           %#3
%          { name }                                  %#4
%          { caption}                                %#5
\def\DFRAME#1#2#3#4#5{%
 \begin{center}
     \let\QCTOptA\empty
     \let\QCTOptB\empty
     \let\QCBOptA\empty
     \let\QCBOptB\empty
     \ifOverFrame 
        #5\QCTOptA\par
     \fi
     \GRAPHIC{#4}{#3}{#1}{#2}{\z@}
     \ifUnderFrame 
        \nobreak\par #5\QCBOptA
     \fi
 \end{center}%
 }%
%
%Macro for Floating graphic object
%   \FFRAME{ framedata f|i tbph x F|T }              %#1
%          { contentswidth (scalar)  }               %#2
%          { contentsheight (scalar) }               %#3
%          { caption }                               %#4
%          { label }                                 %#5
%          { draft name }                            %#6
%          { body }                                  %#7
\def\FFRAME#1#2#3#4#5#6#7{%
 \begin{figure}[#1]%
  \let\QCTOptA\empty
  \let\QCTOptB\empty
  \let\QCBOptA\empty
  \let\QCBOptB\empty
  \ifOverFrame
    #4
    \ifx\QCTOptA\empty
    \else
      \ifx\QCTOptB\empty
        \caption{\QCTOptA}%
      \else
        \caption[\QCTOptB]{\QCTOptA}%
      \fi
    \fi
    \ifUnderFrame\else
      \label{#5}%
    \fi
  \else
    \UnderFrametrue%
  \fi
  \begin{center}\GRAPHIC{#7}{#6}{#2}{#3}{\z@}\end{center}%
  \ifUnderFrame
    #4
    \ifx\QCBOptA\empty
      \caption{}%
    \else
      \ifx\QCBOptB\empty
        \caption{\QCBOptA}%
      \else
        \caption[\QCBOptB]{\QCBOptA}%
      \fi
    \fi
    \label{#5}%
  \fi
  \end{figure}%
 }%
%
%
%    \FRAME{ framedata f|i tbph x F|T }              %#1
%          { contentswidth (scalar)  }               %#2
%          { contentsheight (scalar) }               %#3
%          { vertical shift when in-line (scalar) }  %#4
%          { caption }                               %#5
%          { label }                                 %#6
%          { name }                                  %#7
%          { body }                                  %#8
%
%    framedata is a string which can contain the following
%    characters: idftbphxFT
%    Their meaning is as follows:
%             i, d or f : in-line, display, or floating
%             t,b,p,h   : LaTeX floating placement options
%             x         : fit contents box to contents
%             F or T    : Figure or Table. 
%                         Later this can expand
%                         to a more general float class.
%
%
\newcount\dispkind%

\def\makeactives{
  \catcode`\"=\active
  \catcode`\;=\active
  \catcode`\:=\active
  \catcode`\'=\active
  \catcode`\~=\active
}
\bgroup
   \makeactives
   \gdef\activesoff{%
      \def"{\string"}
      \def;{\string;}
      \def:{\string:}
      \def'{\string'}
      \def~{\string~}
      %\bbl@deactivate{"}%
      %\bbl@deactivate{;}%
      %\bbl@deactivate{:}%
      %\bbl@deactivate{'}%
    }
\egroup

\def\FRAME#1#2#3#4#5#6#7#8{%
 \bgroup
 \@ifundefined{bbl@deactivate}{}{\activesoff}
 \ifnum\draft=\@ne
   \wasdrafttrue
 \else
   \wasdraftfalse%
 \fi
 \def\LaTeXparams{}%
 \dispkind=\z@
 \def\LaTeXparams{}%
 \doFRAMEparams{#1}%
 \ifnum\dispkind=\z@\IFRAME{#2}{#3}{#4}{#7}{#8}{#5}\else
  \ifnum\dispkind=\@ne\DFRAME{#2}{#3}{#7}{#8}{#5}\else
   \ifnum\dispkind=\tw@
    \edef\@tempa{\noexpand\FFRAME{\LaTeXparams}}%
    \@tempa{#2}{#3}{#5}{#6}{#7}{#8}%
    \fi
   \fi
  \fi
  \ifwasdraft\draft=1\else\draft=0\fi{}%
  \egroup
 }%
%
% This macro added to let SW gobble a parameter that
% should not be passed on and expanded. 

\def\TEXUX#1{"texux"}

%
% Macros for text attributes:
%
\def\BF#1{{\bf {#1}}}%
\def\NEG#1{\leavevmode\hbox{\rlap{\thinspace/}{$#1$}}}%
%
%%%%%%%%%%%%%%%%%%%%%%%%%%%%%%%%%%%%%%%%%%%%%%%%%%%%%%%%%%%%%%%%%%%%%%%%
%
%
% macros for user - defined functions
\def\func#1{\mathop{\rm #1}}%
\def\limfunc#1{\mathop{\rm #1}}%

%
% miscellaneous 
%\long\def\QQQ#1#2{}%
\long\def\QQQ#1#2{%
     \long\expandafter\def\csname#1\endcsname{#2}}%
%\def\QTP#1{}% JCS - this was changed becuase style editor will define QTP
\@ifundefined{QTP}{\def\QTP#1{}}{}
\@ifundefined{QEXCLUDE}{\def\QEXCLUDE#1{}}{}
%\@ifundefined{Qcb}{\def\Qcb#1{#1}}{}
%\@ifundefined{Qct}{\def\Qct#1{#1}}{}
\@ifundefined{Qlb}{\def\Qlb#1{#1}}{}
\@ifundefined{Qlt}{\def\Qlt#1{#1}}{}
\def\QWE{}%
\long\def\QQA#1#2{}%
%\def\QTR#1#2{{\em #2}}% Always \em!!!
%\def\QTR#1#2{\mbox{\begin{#1}#2\end{#1}}}%cb%%%
\def\QTR#1#2{{\csname#1\endcsname #2}}%(gp) Is this the best?
\long\def\TeXButton#1#2{#2}%
\long\def\QSubDoc#1#2{#2}%
\def\EXPAND#1[#2]#3{}%
\def\NOEXPAND#1[#2]#3{}%
\def\PROTECTED{}%
\def\LaTeXparent#1{}%
\def\ChildStyles#1{}%
\def\ChildDefaults#1{}%
\def\QTagDef#1#2#3{}%
%
% Macros for style editor docs
\@ifundefined{StyleEditBeginDoc}{\def\StyleEditBeginDoc{\relax}}{}
%
% Macros for footnotes
\def\QQfnmark#1{\footnotemark}
\def\QQfntext#1#2{\addtocounter{footnote}{#1}\footnotetext{#2}}
%
% Macros for indexing.
\def\MAKEINDEX{\makeatletter\input gnuindex.sty\makeatother\makeindex}%	
\@ifundefined{INDEX}{\def\INDEX#1#2{}{}}{}%
\@ifundefined{SUBINDEX}{\def\SUBINDEX#1#2#3{}{}{}}{}%
\@ifundefined{initial}%  
   {\def\initial#1{\bigbreak{\raggedright\large\bf #1}\kern 2\p@\penalty3000}}%
   {}%
\@ifundefined{entry}{\def\entry#1#2{\item {#1}, #2}}{}%
\@ifundefined{primary}{\def\primary#1{\item {#1}}}{}%
\@ifundefined{secondary}{\def\secondary#1#2{\subitem {#1}, #2}}{}%
%
%
\@ifundefined{ZZZ}{}{\MAKEINDEX\makeatletter}%
%
% Attempts to avoid problems with other styles
\@ifundefined{abstract}{%
 \def\abstract{%
  \if@twocolumn
   \section*{Abstract (Not appropriate in this style!)}%
   \else \small 
   \begin{center}{\bf Abstract\vspace{-.5em}\vspace{\z@}}\end{center}%
   \quotation 
   \fi
  }%
 }{%
 }%
\@ifundefined{endabstract}{\def\endabstract
  {\if@twocolumn\else\endquotation\fi}}{}%
\@ifundefined{maketitle}{\def\maketitle#1{}}{}%
\@ifundefined{affiliation}{\def\affiliation#1{}}{}%
\@ifundefined{proof}{\def\proof{\noindent{\bfseries Proof. }}}{}%
\@ifundefined{endproof}{\def\endproof{\mbox{\ \rule{.1in}{.1in}}}}{}%
\@ifundefined{newfield}{\def\newfield#1#2{}}{}%
\@ifundefined{chapter}{\def\chapter#1{\par(Chapter head:)#1\par }%
 \newcount\c@chapter}{}%
\@ifundefined{part}{\def\part#1{\par(Part head:)#1\par }}{}%
\@ifundefined{section}{\def\section#1{\par(Section head:)#1\par }}{}%
\@ifundefined{subsection}{\def\subsection#1%
 {\par(Subsection head:)#1\par }}{}%
\@ifundefined{subsubsection}{\def\subsubsection#1%
 {\par(Subsubsection head:)#1\par }}{}%
\@ifundefined{paragraph}{\def\paragraph#1%
 {\par(Subsubsubsection head:)#1\par }}{}%
\@ifundefined{subparagraph}{\def\subparagraph#1%
 {\par(Subsubsubsubsection head:)#1\par }}{}%
%%%%%%%%%%%%%%%%%%%%%%%%%%%%%%%%%%%%%%%%%%%%%%%%%%%%%%%%%%%%%%%%%%%%%%%%
% These symbols are not recognized by LaTeX
\@ifundefined{therefore}{\def\therefore{}}{}%
\@ifundefined{backepsilon}{\def\backepsilon{}}{}%
\@ifundefined{yen}{\def\yen{\hbox{\rm\rlap=Y}}}{}%
\@ifundefined{registered}{%
   \def\registered{\relax\ifmmode{}\r@gistered
                    \else$\m@th\r@gistered$\fi}%
 \def\r@gistered{^{\ooalign
  {\hfil\raise.07ex\hbox{$\scriptstyle\rm\text{R}$}\hfil\crcr
  \mathhexbox20D}}}}{}%
\@ifundefined{Eth}{\def\Eth{}}{}%
\@ifundefined{eth}{\def\eth{}}{}%
\@ifundefined{Thorn}{\def\Thorn{}}{}%
\@ifundefined{thorn}{\def\thorn{}}{}%
% A macro to allow any symbol that requires math to appear in text
\def\TEXTsymbol#1{\mbox{$#1$}}%
\@ifundefined{degree}{\def\degree{{}^{\circ}}}{}%
%
% macros for T3TeX files
\newdimen\theight
\def\Column{%
 \vadjust{\setbox\z@=\hbox{\scriptsize\quad\quad tcol}%
  \theight=\ht\z@\advance\theight by \dp\z@\advance\theight by \lineskip
  \kern -\theight \vbox to \theight{%
   \rightline{\rlap{\box\z@}}%
   \vss
   }%
  }%
 }%
%
\def\qed{%
 \ifhmode\unskip\nobreak\fi\ifmmode\ifinner\else\hskip5\p@\fi\fi
 \hbox{\hskip5\p@\vrule width4\p@ height6\p@ depth1.5\p@\hskip\p@}%
 }%
%
\def\cents{\hbox{\rm\rlap/c}}%
\def\miss{\hbox{\vrule height2\p@ width 2\p@ depth\z@}}%
%\def\miss{\hbox{.}}%        %another possibility 
%
\def\vvert{\Vert}%           %always translated to \left| or \right|
%
\def\tcol#1{{\baselineskip=6\p@ \vcenter{#1}} \Column}  %
%
\def\dB{\hbox{{}}}%                 %dummy entry in column 
\def\mB#1{\hbox{$#1$}}%             %column entry
\def\nB#1{\hbox{#1}}%               %column entry (not math)
%
%\newcount\notenumber
%\def\clearnotenumber{\notenumber=0}
%\def\note{\global\advance\notenumber by 1
% \footnote{$^{\the\notenumber}$}}
%\def\note{\global\advance\notenumber by 1
\def\note{$^{\dag}}%
%
%

\def\newfmtname{LaTeX2e}
\def\chkcompat{%
   \if@compatibility
   \else
     \usepackage{latexsym}
   \fi
}

\ifx\fmtname\newfmtname
  \DeclareOldFontCommand{\rm}{\normalfont\rmfamily}{\mathrm}
  \DeclareOldFontCommand{\sf}{\normalfont\sffamily}{\mathsf}
  \DeclareOldFontCommand{\tt}{\normalfont\ttfamily}{\mathtt}
  \DeclareOldFontCommand{\bf}{\normalfont\bfseries}{\mathbf}
  \DeclareOldFontCommand{\it}{\normalfont\itshape}{\mathit}
  \DeclareOldFontCommand{\sl}{\normalfont\slshape}{\@nomath\sl}
  \DeclareOldFontCommand{\sc}{\normalfont\scshape}{\@nomath\sc}
  \chkcompat
\fi

%
% Greek bold macros
% Redefine all of the math symbols 
% which might be bolded	 - there are 
% probably others to add to this list

\def\alpha{\Greekmath 010B }%
\def\beta{\Greekmath 010C }%
\def\gamma{\Greekmath 010D }%
\def\delta{\Greekmath 010E }%
\def\epsilon{\Greekmath 010F }%
\def\zeta{\Greekmath 0110 }%
\def\eta{\Greekmath 0111 }%
\def\theta{\Greekmath 0112 }%
\def\iota{\Greekmath 0113 }%
\def\kappa{\Greekmath 0114 }%
\def\lambda{\Greekmath 0115 }%
\def\mu{\Greekmath 0116 }%
\def\nu{\Greekmath 0117 }%
\def\xi{\Greekmath 0118 }%
\def\pi{\Greekmath 0119 }%
\def\rho{\Greekmath 011A }%
\def\sigma{\Greekmath 011B }%
\def\tau{\Greekmath 011C }%
\def\upsilon{\Greekmath 011D }%
\def\phi{\Greekmath 011E }%
\def\chi{\Greekmath 011F }%
\def\psi{\Greekmath 0120 }%
\def\omega{\Greekmath 0121 }%
\def\varepsilon{\Greekmath 0122 }%
\def\vartheta{\Greekmath 0123 }%
\def\varpi{\Greekmath 0124 }%
\def\varrho{\Greekmath 0125 }%
\def\varsigma{\Greekmath 0126 }%
\def\varphi{\Greekmath 0127 }%

\def\nabla{\Greekmath 0272 }
\def\FindBoldGroup{%
   {\setbox0=\hbox{$\mathbf{x\global\edef\theboldgroup{\the\mathgroup}}$}}%
}

\def\Greekmath#1#2#3#4{%
    \if@compatibility
        \ifnum\mathgroup=\symbold
           \mathchoice{\mbox{\boldmath$\displaystyle\mathchar"#1#2#3#4$}}%
                      {\mbox{\boldmath$\textstyle\mathchar"#1#2#3#4$}}%
                      {\mbox{\boldmath$\scriptstyle\mathchar"#1#2#3#4$}}%
                      {\mbox{\boldmath$\scriptscriptstyle\mathchar"#1#2#3#4$}}%
        \else
           \mathchar"#1#2#3#4% 
        \fi 
    \else 
        \FindBoldGroup
        \ifnum\mathgroup=\theboldgroup % For 2e
           \mathchoice{\mbox{\boldmath$\displaystyle\mathchar"#1#2#3#4$}}%
                      {\mbox{\boldmath$\textstyle\mathchar"#1#2#3#4$}}%
                      {\mbox{\boldmath$\scriptstyle\mathchar"#1#2#3#4$}}%
                      {\mbox{\boldmath$\scriptscriptstyle\mathchar"#1#2#3#4$}}%
        \else
           \mathchar"#1#2#3#4% 
        \fi     	    
	  \fi}

\newif\ifGreekBold  \GreekBoldfalse
\let\SAVEPBF=\pbf
\def\pbf{\GreekBoldtrue\SAVEPBF}%
%

\@ifundefined{theorem}{\newtheorem{theorem}{Theorem}}{}
\@ifundefined{lemma}{\newtheorem{lemma}[theorem]{Lemma}}{}
\@ifundefined{corollary}{\newtheorem{corollary}[theorem]{Corollary}}{}
\@ifundefined{conjecture}{\newtheorem{conjecture}[theorem]{Conjecture}}{}
\@ifundefined{proposition}{\newtheorem{proposition}[theorem]{Proposition}}{}
\@ifundefined{axiom}{\newtheorem{axiom}{Axiom}}{}
\@ifundefined{remark}{\newtheorem{remark}{Remark}}{}
\@ifundefined{example}{\newtheorem{example}{Example}}{}
\@ifundefined{exercise}{\newtheorem{exercise}{Exercise}}{}
\@ifundefined{definition}{\newtheorem{definition}{Definition}}{}


\@ifundefined{mathletters}{%
  %\def\theequation{\arabic{equation}}
  \newcounter{equationnumber}  
  \def\mathletters{%
     \addtocounter{equation}{1}
     \edef\@currentlabel{\theequation}%
     \setcounter{equationnumber}{\c@equation}
     \setcounter{equation}{0}%
     \edef\theequation{\@currentlabel\noexpand\alph{equation}}%
  }
  \def\endmathletters{%
     \setcounter{equation}{\value{equationnumber}}%
  }
}{}

%Logos
\@ifundefined{BibTeX}{%
    \def\BibTeX{{\rm B\kern-.05em{\sc i\kern-.025em b}\kern-.08em
                 T\kern-.1667em\lower.7ex\hbox{E}\kern-.125emX}}}{}%
\@ifundefined{AmS}%
    {\def\AmS{{\protect\usefont{OMS}{cmsy}{m}{n}%
                A\kern-.1667em\lower.5ex\hbox{M}\kern-.125emS}}}{}%
\@ifundefined{AmSTeX}{\def\AmSTeX{\protect\AmS-\protect\TeX\@}}{}%
%

%%%%%%%%%%%%%%%%%%%%%%%%%%%%%%%%%%%%%%%%%%%%%%%%%%%%%%%%%%%%%%%%%%%%%%%
% NOTE: The rest of this file is read only if amstex has not been
% loaded.  This section is used to define amstex constructs in the
% event they have not been defined.
%
%
\ifx\ds@amstex\relax
   \message{amstex already loaded}\makeatother\endinput% 2.09 compatability
\else
   \@ifpackageloaded{amstex}%
      {\message{amstex already loaded}\makeatother\endinput}
      {}
   \@ifpackageloaded{amsgen}%
      {\message{amsgen already loaded}\makeatother\endinput}
      {}
\fi
%%%%%%%%%%%%%%%%%%%%%%%%%%%%%%%%%%%%%%%%%%%%%%%%%%%%%%%%%%%%%%%%%%%%%%%%
%%
%
%
%  Macros to define some AMS LaTeX constructs when 
%  AMS LaTeX has not been loaded
% 
% These macros are copied from the AMS-TeX package for doing
% multiple integrals.
%
\let\DOTSI\relax
\def\RIfM@{\relax\ifmmode}%
\def\FN@{\futurelet\next}%
\newcount\intno@
\def\iint{\DOTSI\intno@\tw@\FN@\ints@}%
\def\iiint{\DOTSI\intno@\thr@@\FN@\ints@}%
\def\iiiint{\DOTSI\intno@4 \FN@\ints@}%
\def\idotsint{\DOTSI\intno@\z@\FN@\ints@}%
\def\ints@{\findlimits@\ints@@}%
\newif\iflimtoken@
\newif\iflimits@
\def\findlimits@{\limtoken@true\ifx\next\limits\limits@true
 \else\ifx\next\nolimits\limits@false\else
 \limtoken@false\ifx\ilimits@\nolimits\limits@false\else
 \ifinner\limits@false\else\limits@true\fi\fi\fi\fi}%
\def\multint@{\int\ifnum\intno@=\z@\intdots@                          %1
 \else\intkern@\fi                                                    %2
 \ifnum\intno@>\tw@\int\intkern@\fi                                   %3
 \ifnum\intno@>\thr@@\int\intkern@\fi                                 %4
 \int}%                                                               %5
\def\multintlimits@{\intop\ifnum\intno@=\z@\intdots@\else\intkern@\fi
 \ifnum\intno@>\tw@\intop\intkern@\fi
 \ifnum\intno@>\thr@@\intop\intkern@\fi\intop}%
\def\intic@{%
    \mathchoice{\hskip.5em}{\hskip.4em}{\hskip.4em}{\hskip.4em}}%
\def\negintic@{\mathchoice
 {\hskip-.5em}{\hskip-.4em}{\hskip-.4em}{\hskip-.4em}}%
\def\ints@@{\iflimtoken@                                              %1
 \def\ints@@@{\iflimits@\negintic@
   \mathop{\intic@\multintlimits@}\limits                             %2
  \else\multint@\nolimits\fi                                          %3
  \eat@}%                                                             %4
 \else                                                                %5
 \def\ints@@@{\iflimits@\negintic@
  \mathop{\intic@\multintlimits@}\limits\else
  \multint@\nolimits\fi}\fi\ints@@@}%
\def\intkern@{\mathchoice{\!\!\!}{\!\!}{\!\!}{\!\!}}%
\def\plaincdots@{\mathinner{\cdotp\cdotp\cdotp}}%
\def\intdots@{\mathchoice{\plaincdots@}%
 {{\cdotp}\mkern1.5mu{\cdotp}\mkern1.5mu{\cdotp}}%
 {{\cdotp}\mkern1mu{\cdotp}\mkern1mu{\cdotp}}%
 {{\cdotp}\mkern1mu{\cdotp}\mkern1mu{\cdotp}}}%
%
%
%  These macros are for doing the AMS \text{} construct
%
\def\RIfM@{\relax\protect\ifmmode}
\def\text{\RIfM@\expandafter\text@\else\expandafter\mbox\fi}
\let\nfss@text\text
\def\text@#1{\mathchoice
   {\textdef@\displaystyle\f@size{#1}}%
   {\textdef@\textstyle\tf@size{\firstchoice@false #1}}%
   {\textdef@\textstyle\sf@size{\firstchoice@false #1}}%
   {\textdef@\textstyle \ssf@size{\firstchoice@false #1}}%
   \glb@settings}

\def\textdef@#1#2#3{\hbox{{%
                    \everymath{#1}%
                    \let\f@size#2\selectfont
                    #3}}}
\newif\iffirstchoice@
\firstchoice@true
%
%    Old Scheme for \text
%
%\def\rmfam{\z@}%
%\newif\iffirstchoice@
%\firstchoice@true
%\def\textfonti{\the\textfont\@ne}%
%\def\textfontii{\the\textfont\tw@}%
%\def\text{\RIfM@\expandafter\text@\else\expandafter\text@@\fi}%
%\def\text@@#1{\leavevmode\hbox{#1}}%
%\def\text@#1{\mathchoice
% {\hbox{\everymath{\displaystyle}\def\textfonti{\the\textfont\@ne}%
%  \def\textfontii{\the\textfont\tw@}\textdef@@ T#1}}%
% {\hbox{\firstchoice@false
%  \everymath{\textstyle}\def\textfonti{\the\textfont\@ne}%
%  \def\textfontii{\the\textfont\tw@}\textdef@@ T#1}}%
% {\hbox{\firstchoice@false
%  \everymath{\scriptstyle}\def\textfonti{\the\scriptfont\@ne}%
%  \def\textfontii{\the\scriptfont\tw@}\textdef@@ S\rm#1}}%
% {\hbox{\firstchoice@false
%  \everymath{\scriptscriptstyle}\def\textfonti
%  {\the\scriptscriptfont\@ne}%
%  \def\textfontii{\the\scriptscriptfont\tw@}\textdef@@ s\rm#1}}}%
%\def\textdef@@#1{\textdef@#1\rm\textdef@#1\bf\textdef@#1\sl
%    \textdef@#1\it}%
%\def\DN@{\def\next@}%
%\def\eat@#1{}%
%\def\textdef@#1#2{%
% \DN@{\csname\expandafter\eat@\string#2fam\endcsname}%
% \if S#1\edef#2{\the\scriptfont\next@\relax}%
% \else\if s#1\edef#2{\the\scriptscriptfont\next@\relax}%
% \else\edef#2{\the\textfont\next@\relax}\fi\fi}%
%
%
%These are the AMS constructs for multiline limits.
%
\def\Let@{\relax\iffalse{\fi\let\\=\cr\iffalse}\fi}%
\def\vspace@{\def\vspace##1{\crcr\noalign{\vskip##1\relax}}}%
\def\multilimits@{\bgroup\vspace@\Let@
 \baselineskip\fontdimen10 \scriptfont\tw@
 \advance\baselineskip\fontdimen12 \scriptfont\tw@
 \lineskip\thr@@\fontdimen8 \scriptfont\thr@@
 \lineskiplimit\lineskip
 \vbox\bgroup\ialign\bgroup\hfil$\m@th\scriptstyle{##}$\hfil\crcr}%
\def\Sb{_\multilimits@}%
\def\endSb{\crcr\egroup\egroup\egroup}%
\def\Sp{^\multilimits@}%
\let\endSp\endSb
%
%
%These are AMS constructs for horizontal arrows
%
\newdimen\ex@
\ex@.2326ex
\def\rightarrowfill@#1{$#1\m@th\mathord-\mkern-6mu\cleaders
 \hbox{$#1\mkern-2mu\mathord-\mkern-2mu$}\hfill
 \mkern-6mu\mathord\rightarrow$}%
\def\leftarrowfill@#1{$#1\m@th\mathord\leftarrow\mkern-6mu\cleaders
 \hbox{$#1\mkern-2mu\mathord-\mkern-2mu$}\hfill\mkern-6mu\mathord-$}%
\def\leftrightarrowfill@#1{$#1\m@th\mathord\leftarrow
\mkern-6mu\cleaders
 \hbox{$#1\mkern-2mu\mathord-\mkern-2mu$}\hfill
 \mkern-6mu\mathord\rightarrow$}%
\def\overrightarrow{\mathpalette\overrightarrow@}%
\def\overrightarrow@#1#2{\vbox{\ialign{##\crcr\rightarrowfill@#1\crcr
 \noalign{\kern-\ex@\nointerlineskip}$\m@th\hfil#1#2\hfil$\crcr}}}%
\let\overarrow\overrightarrow
\def\overleftarrow{\mathpalette\overleftarrow@}%
\def\overleftarrow@#1#2{\vbox{\ialign{##\crcr\leftarrowfill@#1\crcr
 \noalign{\kern-\ex@\nointerlineskip}$\m@th\hfil#1#2\hfil$\crcr}}}%
\def\overleftrightarrow{\mathpalette\overleftrightarrow@}%
\def\overleftrightarrow@#1#2{\vbox{\ialign{##\crcr
   \leftrightarrowfill@#1\crcr
 \noalign{\kern-\ex@\nointerlineskip}$\m@th\hfil#1#2\hfil$\crcr}}}%
\def\underrightarrow{\mathpalette\underrightarrow@}%
\def\underrightarrow@#1#2{\vtop{\ialign{##\crcr$\m@th\hfil#1#2\hfil
  $\crcr\noalign{\nointerlineskip}\rightarrowfill@#1\crcr}}}%
\let\underarrow\underrightarrow
\def\underleftarrow{\mathpalette\underleftarrow@}%
\def\underleftarrow@#1#2{\vtop{\ialign{##\crcr$\m@th\hfil#1#2\hfil
  $\crcr\noalign{\nointerlineskip}\leftarrowfill@#1\crcr}}}%
\def\underleftrightarrow{\mathpalette\underleftrightarrow@}%
\def\underleftrightarrow@#1#2{\vtop{\ialign{##\crcr$\m@th
  \hfil#1#2\hfil$\crcr
 \noalign{\nointerlineskip}\leftrightarrowfill@#1\crcr}}}%
%%%%%%%%%%%%%%%%%%%%%

% 94.0815 by Jon:

\def\qopnamewl@#1{\mathop{\operator@font#1}\nlimits@}
\let\nlimits@\displaylimits
\def\setboxz@h{\setbox\z@\hbox}


\def\varlim@#1#2{\mathop{\vtop{\ialign{##\crcr
 \hfil$#1\m@th\operator@font lim$\hfil\crcr
 \noalign{\nointerlineskip}#2#1\crcr
 \noalign{\nointerlineskip\kern-\ex@}\crcr}}}}

 \def\rightarrowfill@#1{\m@th\setboxz@h{$#1-$}\ht\z@\z@
  $#1\copy\z@\mkern-6mu\cleaders
  \hbox{$#1\mkern-2mu\box\z@\mkern-2mu$}\hfill
  \mkern-6mu\mathord\rightarrow$}
\def\leftarrowfill@#1{\m@th\setboxz@h{$#1-$}\ht\z@\z@
  $#1\mathord\leftarrow\mkern-6mu\cleaders
  \hbox{$#1\mkern-2mu\copy\z@\mkern-2mu$}\hfill
  \mkern-6mu\box\z@$}


\def\projlim{\qopnamewl@{proj\,lim}}
\def\injlim{\qopnamewl@{inj\,lim}}
\def\varinjlim{\mathpalette\varlim@\rightarrowfill@}
\def\varprojlim{\mathpalette\varlim@\leftarrowfill@}
\def\varliminf{\mathpalette\varliminf@{}}
\def\varliminf@#1{\mathop{\underline{\vrule\@depth.2\ex@\@width\z@
   \hbox{$#1\m@th\operator@font lim$}}}}
\def\varlimsup{\mathpalette\varlimsup@{}}
\def\varlimsup@#1{\mathop{\overline
  {\hbox{$#1\m@th\operator@font lim$}}}}

%
%%%%%%%%%%%%%%%%%%%%%%%%%%%%%%%%%%%%%%%%%%%%%%%%%%%%%%%%%%%%%%%%%%%%%
%
\def\tfrac#1#2{{\textstyle {#1 \over #2}}}%
\def\dfrac#1#2{{\displaystyle {#1 \over #2}}}%
\def\binom#1#2{{#1 \choose #2}}%
\def\tbinom#1#2{{\textstyle {#1 \choose #2}}}%
\def\dbinom#1#2{{\displaystyle {#1 \choose #2}}}%
\def\QATOP#1#2{{#1 \atop #2}}%
\def\QTATOP#1#2{{\textstyle {#1 \atop #2}}}%
\def\QDATOP#1#2{{\displaystyle {#1 \atop #2}}}%
\def\QABOVE#1#2#3{{#2 \above#1 #3}}%
\def\QTABOVE#1#2#3{{\textstyle {#2 \above#1 #3}}}%
\def\QDABOVE#1#2#3{{\displaystyle {#2 \above#1 #3}}}%
\def\QOVERD#1#2#3#4{{#3 \overwithdelims#1#2 #4}}%
\def\QTOVERD#1#2#3#4{{\textstyle {#3 \overwithdelims#1#2 #4}}}%
\def\QDOVERD#1#2#3#4{{\displaystyle {#3 \overwithdelims#1#2 #4}}}%
\def\QATOPD#1#2#3#4{{#3 \atopwithdelims#1#2 #4}}%
\def\QTATOPD#1#2#3#4{{\textstyle {#3 \atopwithdelims#1#2 #4}}}%
\def\QDATOPD#1#2#3#4{{\displaystyle {#3 \atopwithdelims#1#2 #4}}}%
\def\QABOVED#1#2#3#4#5{{#4 \abovewithdelims#1#2#3 #5}}%
\def\QTABOVED#1#2#3#4#5{{\textstyle 
   {#4 \abovewithdelims#1#2#3 #5}}}%
\def\QDABOVED#1#2#3#4#5{{\displaystyle 
   {#4 \abovewithdelims#1#2#3 #5}}}%
%
% Macros for text size operators:

%JCS - added braces and \mathop around \displaystyle\int, etc.
%
\def\tint{\mathop{\textstyle \int}}%
\def\tiint{\mathop{\textstyle \iint }}%
\def\tiiint{\mathop{\textstyle \iiint }}%
\def\tiiiint{\mathop{\textstyle \iiiint }}%
\def\tidotsint{\mathop{\textstyle \idotsint }}%
\def\toint{\mathop{\textstyle \oint}}%
\def\tsum{\mathop{\textstyle \sum }}%
\def\tprod{\mathop{\textstyle \prod }}%
\def\tbigcap{\mathop{\textstyle \bigcap }}%
\def\tbigwedge{\mathop{\textstyle \bigwedge }}%
\def\tbigoplus{\mathop{\textstyle \bigoplus }}%
\def\tbigodot{\mathop{\textstyle \bigodot }}%
\def\tbigsqcup{\mathop{\textstyle \bigsqcup }}%
\def\tcoprod{\mathop{\textstyle \coprod }}%
\def\tbigcup{\mathop{\textstyle \bigcup }}%
\def\tbigvee{\mathop{\textstyle \bigvee }}%
\def\tbigotimes{\mathop{\textstyle \bigotimes }}%
\def\tbiguplus{\mathop{\textstyle \biguplus }}%
%
%
%Macros for display size operators:
%

\def\dint{\mathop{\displaystyle \int}}%
\def\diint{\mathop{\displaystyle \iint }}%
\def\diiint{\mathop{\displaystyle \iiint }}%
\def\diiiint{\mathop{\displaystyle \iiiint }}%
\def\didotsint{\mathop{\displaystyle \idotsint }}%
\def\doint{\mathop{\displaystyle \oint}}%
\def\dsum{\mathop{\displaystyle \sum }}%
\def\dprod{\mathop{\displaystyle \prod }}%
\def\dbigcap{\mathop{\displaystyle \bigcap }}%
\def\dbigwedge{\mathop{\displaystyle \bigwedge }}%
\def\dbigoplus{\mathop{\displaystyle \bigoplus }}%
\def\dbigodot{\mathop{\displaystyle \bigodot }}%
\def\dbigsqcup{\mathop{\displaystyle \bigsqcup }}%
\def\dcoprod{\mathop{\displaystyle \coprod }}%
\def\dbigcup{\mathop{\displaystyle \bigcup }}%
\def\dbigvee{\mathop{\displaystyle \bigvee }}%
\def\dbigotimes{\mathop{\displaystyle \bigotimes }}%
\def\dbiguplus{\mathop{\displaystyle \biguplus }}%
%
%Companion to stackrel
\def\stackunder#1#2{\mathrel{\mathop{#2}\limits_{#1}}}%
%
%
% These are AMS environments that will be defined to
% be verbatims if amstex has not actually been 
% loaded
%
%
\begingroup \catcode `|=0 \catcode `[= 1
\catcode`]=2 \catcode `\{=12 \catcode `\}=12
\catcode`\\=12 
|gdef|@alignverbatim#1\end{align}[#1|end[align]]
|gdef|@salignverbatim#1\end{align*}[#1|end[align*]]

|gdef|@alignatverbatim#1\end{alignat}[#1|end[alignat]]
|gdef|@salignatverbatim#1\end{alignat*}[#1|end[alignat*]]

|gdef|@xalignatverbatim#1\end{xalignat}[#1|end[xalignat]]
|gdef|@sxalignatverbatim#1\end{xalignat*}[#1|end[xalignat*]]

|gdef|@gatherverbatim#1\end{gather}[#1|end[gather]]
|gdef|@sgatherverbatim#1\end{gather*}[#1|end[gather*]]

|gdef|@gatherverbatim#1\end{gather}[#1|end[gather]]
|gdef|@sgatherverbatim#1\end{gather*}[#1|end[gather*]]


|gdef|@multilineverbatim#1\end{multiline}[#1|end[multiline]]
|gdef|@smultilineverbatim#1\end{multiline*}[#1|end[multiline*]]

|gdef|@arraxverbatim#1\end{arrax}[#1|end[arrax]]
|gdef|@sarraxverbatim#1\end{arrax*}[#1|end[arrax*]]

|gdef|@tabulaxverbatim#1\end{tabulax}[#1|end[tabulax]]
|gdef|@stabulaxverbatim#1\end{tabulax*}[#1|end[tabulax*]]


|endgroup
  

  
\def\align{\@verbatim \frenchspacing\@vobeyspaces \@alignverbatim
You are using the "align" environment in a style in which it is not defined.}
\let\endalign=\endtrivlist
 
\@namedef{align*}{\@verbatim\@salignverbatim
You are using the "align*" environment in a style in which it is not defined.}
\expandafter\let\csname endalign*\endcsname =\endtrivlist




\def\alignat{\@verbatim \frenchspacing\@vobeyspaces \@alignatverbatim
You are using the "alignat" environment in a style in which it is not defined.}
\let\endalignat=\endtrivlist
 
\@namedef{alignat*}{\@verbatim\@salignatverbatim
You are using the "alignat*" environment in a style in which it is not defined.}
\expandafter\let\csname endalignat*\endcsname =\endtrivlist




\def\xalignat{\@verbatim \frenchspacing\@vobeyspaces \@xalignatverbatim
You are using the "xalignat" environment in a style in which it is not defined.}
\let\endxalignat=\endtrivlist
 
\@namedef{xalignat*}{\@verbatim\@sxalignatverbatim
You are using the "xalignat*" environment in a style in which it is not defined.}
\expandafter\let\csname endxalignat*\endcsname =\endtrivlist




\def\gather{\@verbatim \frenchspacing\@vobeyspaces \@gatherverbatim
You are using the "gather" environment in a style in which it is not defined.}
\let\endgather=\endtrivlist
 
\@namedef{gather*}{\@verbatim\@sgatherverbatim
You are using the "gather*" environment in a style in which it is not defined.}
\expandafter\let\csname endgather*\endcsname =\endtrivlist


\def\multiline{\@verbatim \frenchspacing\@vobeyspaces \@multilineverbatim
You are using the "multiline" environment in a style in which it is not defined.}
\let\endmultiline=\endtrivlist
 
\@namedef{multiline*}{\@verbatim\@smultilineverbatim
You are using the "multiline*" environment in a style in which it is not defined.}
\expandafter\let\csname endmultiline*\endcsname =\endtrivlist


\def\arrax{\@verbatim \frenchspacing\@vobeyspaces \@arraxverbatim
You are using a type of "array" construct that is only allowed in AmS-LaTeX.}
\let\endarrax=\endtrivlist

\def\tabulax{\@verbatim \frenchspacing\@vobeyspaces \@tabulaxverbatim
You are using a type of "tabular" construct that is only allowed in AmS-LaTeX.}
\let\endtabulax=\endtrivlist

 
\@namedef{arrax*}{\@verbatim\@sarraxverbatim
You are using a type of "array*" construct that is only allowed in AmS-LaTeX.}
\expandafter\let\csname endarrax*\endcsname =\endtrivlist

\@namedef{tabulax*}{\@verbatim\@stabulaxverbatim
You are using a type of "tabular*" construct that is only allowed in AmS-LaTeX.}
\expandafter\let\csname endtabulax*\endcsname =\endtrivlist

% macro to simulate ams tag construct


% This macro is a fix to eqnarray
\def\@@eqncr{\let\@tempa\relax
    \ifcase\@eqcnt \def\@tempa{& & &}\or \def\@tempa{& &}%
      \else \def\@tempa{&}\fi
     \@tempa
     \if@eqnsw
        \iftag@
           \@taggnum
        \else
           \@eqnnum\stepcounter{equation}%
        \fi
     \fi
     \global\tag@false
     \global\@eqnswtrue
     \global\@eqcnt\z@\cr}


% This macro is a fix to the equation environment
 \def\endequation{%
     \ifmmode\ifinner % FLEQN hack
      \iftag@
        \addtocounter{equation}{-1} % undo the increment made in the begin part
        $\hfil
           \displaywidth\linewidth\@taggnum\egroup \endtrivlist
        \global\tag@false
        \global\@ignoretrue   
      \else
        $\hfil
           \displaywidth\linewidth\@eqnnum\egroup \endtrivlist
        \global\tag@false
        \global\@ignoretrue 
      \fi
     \else   
      \iftag@
        \addtocounter{equation}{-1} % undo the increment made in the begin part
        \eqno \hbox{\@taggnum}
        \global\tag@false%
        $$\global\@ignoretrue
      \else
        \eqno \hbox{\@eqnnum}% $$ BRACE MATCHING HACK
        $$\global\@ignoretrue
      \fi
     \fi\fi
 } 

 \newif\iftag@ \tag@false
 
 \def\tag{\@ifnextchar*{\@tagstar}{\@tag}}
 \def\@tag#1{%
     \global\tag@true
     \global\def\@taggnum{(#1)}}
 \def\@tagstar*#1{%
     \global\tag@true
     \global\def\@taggnum{#1}%  
}

% Do not add anything to the end of this file.  
% The last section of the file is loaded only if 
% amstex has not been.



\makeatother
\endinput

\setlength{\abovecaptionskip}{3pt}
\makeatletter
\def\@biblabel#1{\hspace*{-\labelsep}}
\newdimen\@bibhang \@bibhang=2em
\def\setbibhang#1{\@bibhang=#1}
\renewenvironment{thebibliography}[1]{  \setlength{\labelwidth}{0pt}
  \setlength{\labelsep}{0pt}
  \section*{\refname
     \@mkboth{\uppercase{\refname}}{\uppercase{\refname}}}   \vspace{0em}
   \setlength{\parindent}{0pt}
   \def\newblock{}
   \renewcommand{\bibitem}[2][]{     \if@filesw
       {\let\protect\noexpand\immediate
        \write\@auxout{\string\bibcite{##2}{##1}}}
        \fi\hangindent=\@bibhang\hangafter=1}}
\makeatother

\begin{document}

\title{Estimation of nested, zero-inflated and cross-nested ordered probit
models in STATA\\
\bigskip \bigskip }
\author{David Dale and Andrei Sirchenko}
\date{March 13, 2017}
\maketitle

\begin{abstract}
TBA

\bigskip \bigskip \bigskip \bigskip

\textit{JEL classification:} \bigskip .

\textit{Keywords:} ordinal responses, zero-inflated outcomes, two- and
three-part mixture model, endogenous regime switching.
\end{abstract}

%TCIMACRO{\TeXButton{Onehalf}{\begin{onehalfspace}}}%
%BeginExpansion
\begin{onehalfspace}%
%EndExpansion

\section{Introduction}

TBA...

\section{Econometric framework}

Left out

\section{Stata commands}

\subsection*{Syntax of the cnop, miop, and nop commands}

%\subsubsection{Syntax}

\hangindent=\parindent
\noindent \texttt{cnop $depvar$ $indepvars$ [$if$] [$in$] [, zp($varlist$)
zn($varlist$) infcat($integer$ $0$) correlated cluster($varname$) robust
initial($string$)] }

This command fits a cross-nested ordered probit model with possibly
different sets of covariates for each stage and possibly correlated errors
by maximum likelihood.

\hangindent=\parindent
\noindent \texttt{miop $depvar$ $indepvars$ [$if$] [$in$] [, z ($varlist$)
infcat($integer$ $0$) correlated cluster($varname$) robust initial($string$%
)] }

This command fits a middle-inflated ordered probit model.

\hangindent=\parindent
\noindent \texttt{nop $depvar$ $indepvars$ [$if$] [$in$] [, zp($varlist$) zn(%
$varlist$) infcat($integer$ $0$) correlated cluster($varname$) robust
initial($string$)] }

This command fits a nested ordered probit model.

%\subsubsection{Description}

\subsubsection*{Options}

\begin{tabular}{lp{12cm}}
\textit{options} & Description \\ 
\midrule \texttt{zp($varlist$)} & list of covariates for positive response
in NOP and CNOP models; by default, it equals $indepvars$, the list of
covariates for initial stage \\ 
\texttt{zn($varlist$)} & list of covariates for negative response in NOP and
CNOP models; by default, it equals $indepvars$, the list of covariates for
initial stage \\ 
\texttt{z($varlist$)} & list of covariates for non-zero response in ZIOP
models; by default, it equals $indepvars$, the list of covariates for
initial stage \\ 
\texttt{infcat($integer$)} & value of the response variable that should be
modeled as inflated; by default, it equals 0 \\ 
\texttt{correlated} & flag that errors in the first and second stages may be
correlated, forcing estimation of CNOPc, NOPc or ZIOPc model \\ 
\texttt{robust} & flag that variance-covariance estimator must be robust
(based on ``sandwich``) estimate \\ 
\texttt{cluster($varname$)} & clustering variable for robust variance
estimator \\ 
\texttt{initial($string$)} & whitespace-delimited list of initial parameter
values for estimation, in the following order: $\beta$, $\alpha$, $\gamma^{+}
$, $\mu^{+}$, $\gamma^{-}$, $\mu^{-}$, $\rho^{-}$, $\rho^{+}$%
\end{tabular}

\subsubsection*{Examples}

TBD

\subsubsection*{Stored results}

\texttt{cnop}, \texttt{nop}, and \texttt{miop} store the following in 
\texttt{e()}:

%Scalars

\begin{tabular}{p{3cm}p{12cm}}
\texttt{e(N)} & number of observations%
\end{tabular}

%Macros

\begin{tabular}{p{3cm}p{12cm}}
\texttt{e(cmd)} & \texttt{cnop}, \texttt{nop}, or \texttt{miop}, respectively
\\ 
\texttt{e(depvar)} & dependent variable of regression%
\end{tabular}

%Matrices

\begin{tabular}{p{3cm}p{12cm}}
\texttt{e(b)} & parameters vector \\ 
\texttt{e(V)} & variance-covariance matrix%
\end{tabular}

%Functions

\begin{tabular}{p{3cm}p{12cm}}
\texttt{e(sample)} & marks estimation sample%
\end{tabular}

\subsection*{CNOP postestimation commands}

\subsubsection*{The predict command}

The \texttt{predict} command after \texttt{cnop}, \texttt{nop}, and \texttt{%
miop} estimation commands produces either predicted probabilities or
expected value of the response.

\texttt{predict $varname$ [$if$] [$in$] [, zeroes regime output($string$) at(%
$string$)]}

\texttt{name} is the name of predicted variable, if it is single, or prefix
for names, if there are several predicted variables

\texttt{zeroes} indicates that different types of zeroes (i.e. ``intrinsic
zeroes``, or ``positive zeroes``, or ``negative zeroes``) must be predicted
instead of different response values.

\texttt{regime} indicates that different groups of response (negative,
positive or zero) must be predicted instead of different response values.
This option is ignored if \texttt{zeroes} option is on.

\texttt{output(string)} specifies type of aggregating predicted
probabilities of different response. Possible values are \texttt{mode} and 
\texttt{mean}, for predicting average or most probable outcome, and \texttt{%
cum} for predicting cumulative response probabilities (i.e. $p(y <=-2)$, $%
p(y<=-1)$, $p(y<=0)$ etc.). If not specified, raw response probabilities are
predicted ($p(y=-2)$, $p(y=-1)$, $p(y=0)$ etc.).

\subsubsection*{The cnopmargins command}

\texttt{cnopmargins [, at($string$) nominal($varlist$) zeroes regime]}

This command prints marginal effects for the last estimated CNOP, MIOP or
NOP model, calculated at the specified point, along with confidence
intervals.

\texttt{at(string)} specifies at which point predictions must be calculated.
If at is specified, (as a list of \texttt{varname=value} expressions,
separated by comma), prediction is calculated at this point and posted on
the screen without saving to the dataset. If some covariate names are not
specified, their mean value is taken instead.

\texttt{nominal} is a space-separated list of covariates which should be
considered as nominal; marginal effect is then calculated as difference
between values at 0 and at 1.

\texttt{zeroes} and \texttt{regime} indicate that marginal effects should be
calculated for different zeroes or for groups of response variable, as in 
\texttt{predict} command.

\subsubsection*{The cnopprobabilities command}

\texttt{cnopprobabilities [, at($string$) zeroes regime]}

This command prints predicted probabilities for the last estimated CNOP,
MIOP or NOP model, calculated at the specified point, along with confidence
intervals. The point \texttt{at} is specified like in \texttt{cnopmargins}.

\subsubsection*{The cnopcontrasts command}

\texttt{cnopcontrasts [, at($string$) to($string$) zeroes regime] }

This command prints differences in predicted probabilities for the last
estimated CNOP, MIOP or NOP model, calculated between the specified points,
along with confidence intervals. The points \texttt{at} and \texttt{to} are
specified like \texttt{at} in \texttt{cnopmargins}.

\subsubsection*{Examples}

TBD

\section{Finite sample performance}

We conducted extensive Monte Carlo experiments to illustrate the finite
sample performance of the ML estimators in the proposed models.

\subsection*{Monte Carlo design}

We conducted simulations for six data-generating processes (\textit{dgp):}
NOP, NOPc, ZIOP (MIOP version), ZIOPc (MIOPc version), CNOP, and CNOPc. The
data were generated and then estimated by the same model. For each dgp we
generated samples with 200, 500 and 1000 observations. The number of
replications was 10,000 in each experiment.

Three vectors of covariates $\mathbf{v}_{\mathbf{1}}$, $\mathbf{v}_{\mathbf{2%
}}$ and $\mathbf{v}_{\mathbf{3}}$ were drawn in each replication
as\noindent\ $\mathbf{v}_{\mathbf{1}}\overset{\emph{iid}}{\sim }\emph{Normal}%
(0,1)+2$, $\mathbf{v}_{\mathbf{2}}\overset{\emph{iid}}{\sim }\emph{Normal}%
(0,1$), and $\mathbf{v}_{\mathbf{3}}=-1$ if $\mathbf{w}\leq 0.3$, $0$ if $%
0.3<\mathbf{w}\leq 0.7$, or $1$ if $\mathbf{w}>0.7$, where $\mathbf{w}%
\overset{\emph{iid}}{\sim }\emph{Uniform}[0,1]$. The dependent variable was
generated with five outcome categories: -2, -1, 0, 1 and 2. The values of
the parameters were calibrated to yield on average the following frequencies
of the above outcomes: 7\%, 14\%, 58\%, 14\% and 7\%, respectively. To avoid
the divergence of ML estimates due to the problem of complete separation
(perfect prediction), which could happen if actual number of observations in
some outcome category (specifically, -2 and 2) is very low, the samples with
any category frequency lower than 6\% were re-generated. At each iteration
we checked that there is at least 6\% of observations in each outcome
category. The matrix of the PE, therefore, has $3\times 5=15$ elements;
their values, which depend on the values of the explanatory variables, are
computed at the population medians of the covariates. The observations in
repeated samples were drawn independently. The vectors of disturbance terms
in the latent equations were repeatedly generated as iid\emph{\ }$\emph{%
Normal}(0,1)$ random variables in the case of the NOP, ZIOP and CNOP \textit{%
dgp}. In the case of the NOPc and CNOPc models, the errors $\mathbf{\nu }$
in the inclination equation were generated as iid\emph{\ }$\emph{Normal}(0,1)
$ random variables, but the errors $\mathbf{\varepsilon }^{-}$ and $\mathbf{%
\varepsilon }^{+}$ in the amount equations were drawn so that $(\mathbf{\nu }%
,\mathbf{\varepsilon }^{-})$ and $(\mathbf{\nu },\mathbf{\varepsilon }^{+})$
are the standardized bivariate normal iid random variables with the
correlation coefficients $\rho ^{-}$ and $\rho ^{+},$ respectively. In the
ZIOPc dgp, the errors $\mathbf{\nu }^{0}$ in the participation equation were
generated as IID\emph{\ }$\emph{Normal}(0,1$) random variables, but the
errors $\mathbf{\varepsilon }^{0}$ in the amount equation were drawn so that 
$(\mathbf{\nu }^{0},\mathbf{\varepsilon }^{0})$ are the standardized
bivariate normal iid random variables with the correlation coefficients $%
\rho ^{0}$. The repeated samples were generated for the NOP, NOPc, CNOP and
CNOPc \textit{dgp }with $\mathbf{X=(v_{1}},\mathbf{v_{2}}$), $\mathbf{%
\mathbf{Z}^{-}=(v_{1}},\mathbf{v_{3}}$), $\mathbf{\mathbf{Z}^{+}=(v_{2}},%
\mathbf{v_{3}}$), and for the ZIOP and ZIOPc dgp with $\mathbf{X}^{0}\mathbf{%
=(v}_{1},\mathbf{v}_{3})$, $\mathbf{\mathbf{Z}}^{0}\mathbf{=(v}_{2},\mathbf{v%
}_{3})$. The true values of the simulation parameters are shown in Table \ref%
{tab:params}.

% Table generated by Excel2LaTeX from sheet 'params'

\begin{table}[htbp]
\caption{True values of parameters for simulation}
\label{tab:params}{\tiny \ \centering
\begin{tabular}{lcccccc}
\toprule & NOP & NOPc & ZIOP & ZIOPc & CNOP & CNOPc \\ 
\midrule $\mathbf{\beta }$ & (0.6, 0.4)$^{\prime }$ & (0.6, 0.4)$^{\prime }$
& (0.6, 0.8)$^{\prime }$ & (0.6, 0.8)$^{\prime }$ & (0.6, 0.4)$^{\prime }$ & 
(0.6, 0.4)$^{\prime }$ \\ 
$\mathbf{\alpha }$ & (0.21, 2.19)$^{\prime }$ & (0.21, 2.19)$^{\prime }$ & 
0.45 & 0.45 & (0.9, 1.5)$^{\prime }$ & (0.9, 1.5)$^{\prime }$ \\ 
$\mathbf{\gamma }^{-}$ & (0.2, 0.3)$^{\prime }$ & (0.2, 0.3)$^{\prime }$ & 
&  & (0.2, 0.3)$^{\prime }$ & (0.2, 0.3)$^{\prime }$ \\ 
$\mathbf{\gamma }^{+}$ & (0.3, 0.9)$^{\prime }$ & (0.3, 0.9)$^{\prime }$ & 
&  & (0.3, 0.9)$^{\prime }$ & (0.3, 0.9)$^{\prime }$ \\ 
$\mathbf{\mu }^{-}$ & -0.17 & -0.5 &  &  & (-0.67, 0.36)$^{\prime }$ & 
(-0.88, 0.12)$^{\prime }$ \\ 
$\mathbf{\mu }^{+}$ & 0.68 & 1.31 &  &  & (0.02, 1.28)$^{\prime }$ & (0.49,
1.67)$^{\prime }$ \\ 
$\rho ^{-}$ &  & 0.3 &  &  &  & 0.3 \\ 
$\rho ^{+}$ &  & 0.6 &  &  &  & 0.6 \\ 
$\mathbf{\gamma }^{0}$ &  &  & (0.5, 0.6)$^{\prime }$ & (0.5, 0.6)$^{\prime }
$ &  &  \\ 
$\mathbf{\mu }^{0}$ &  &  & (-1.45, -0.55, 0.75, 1.65)$^{^{\prime }}$ & 
(-1.18,-0.33, 0.9, 1.76)$^{^{\prime }}$ &  &  \\ 
$\rho ^{0}$ &  &  &  & 0.5 &  &  \\ 
\bottomrule &  &  &  &  &  & 
\end{tabular}
}
\end{table}

Table \ref{table:errors} reports the following measures of accuracy computed
for the estimates of the parameters, probabilities and PE: \emph{Bias} ---
the absolute difference between the estimated and true values, devided by
the true value, averaged over all Monte Carlo runs, in percent; \emph{RMSE}
--- the absolute root mean square error of the parameter estimates relative
to their true values, averaged over all replications; \emph{CP} --- the
empirical coverage probability, computed as the percentage of times the
estimated asymptotic 95\% confidence intervals cover the true values. To
measure the accuracy of the estimates of the standard errors, we also
computed the \emph{s.e.} \emph{bias} --- the absolute difference between the
average of the estimated asymptotic standard errors of the estimates and the
standard deviation of the estimates in all replications, in percent. The
above measures of accuracy computed for the estimates of the parameters are
averaged across all parameters, for the estimates of the probabilities ---
averaged across five outcome categories, and for the estimates of the PEs
--- averaged across five outcome categories and across all covariates.

The simulations and estimations were performed using the MATA programming
language. The starting values for $\mathbf{\alpha }$, $\mathbf{\beta }$, $%
\mathbf{\beta }^{0}$, $\mathbf{\mu }^{0}$, $\mathbf{\mu }^{+},$ $\mathbf{%
\gamma }^{+}$, $\mathbf{\mu }^{-}$ and $\mathbf{\gamma }^{-}$ were obtained
using the independent OP estimations of each latent equation. The starting
values for each independent OP model were computed using the linear OLS
estimations. The starting values for $\rho ^{0}$, $\rho ^{-}$ and $\rho ^{+}$
were obtained by maximizing the logarithms of the likelihood functions of
the correlated models holding the other parameters fixed at their estimates
in the corresponding uncorrelated model.

\subsection*{Results of simulations}

For each model specification, bias and RMSE of parameter estimates decrease
as sample size increases. RMSE decreases in most cases faster than
asymptotic rate $\sqrt{n}$. This may be caused by a small number of large
deviations in parameter estimation on small samples.

For most of model pairs and sample sizes, bias and RMSE is slightly higher
for the correlated version. This is expected from a more complex model,
estimated on the same sample size.

Standard error estimates for parameters on average correspond to the actual
standard errors. Large deviations make standard errors estimates biased,
especially on small samples, but this problem rapidly decreases with sample
size. Anyway, rare large deviations do not prevent asymptotic coverage
probabilities of 95\% confidence intervals from being consistent. This means
that confidence intervals for parameter estimates may be used safely.

For estimates of outcome probabilities and marginal effects the situation is
qualitatively and quantitatively similar to estimates of parameters.

In general, results of Monte Carlo simulations show that estimators of CNOP
family are consistent, but should be used carefully on small samples. As a
rule of thumb, we would advise using at least 10 observations per variable
in each outcome class, which corresponds to 1000 observations in our case.

\section{\protect\Large The (correlated) nested ordered probit model}

Left out

\section*{\protect\Large A special case when the CNOP model nests the MIOP
model\ }

Left out

% Need to insert \midrule between parts of table and 4cm width of multirows instead of * manually
% Need \usepackage{booktabs} and \usepackage{multirow}
% Table generated by Excel2LaTeX from sheet 'total'. 

\begin{table}[htbp]
\caption{Monte Carlo simulations for different models and sample sizes}
\label{table:errors}\centering
\begin{tabular}{rrcccccc}
\toprule \toprule Sample size & \multicolumn{1}{c}{DGP:} & NOP & NOPc & ZIOP
& ZIOPc & CNOP & CNOPc \\ 
\multicolumn{8}{c}{Parameters} \\ 
\multicolumn{1}{l}{200} & \multicolumn{1}{l}{\multirow{3}[1]{*}{Bias, \%}} & 
12.8 & 19.5 & 42.3 & 11.3 & 59.4 & 89.4 \\ 
\multicolumn{1}{l}{500} & \multicolumn{1}{l}{} & 4.8 & 12.2 & 3.8 & 3.7 & 
20.9 & 19.6 \\ 
\multicolumn{1}{l}{1000} & \multicolumn{1}{l}{} & 2.2 & 5.2 & 1.8 & 1.7 & 8.9
& 5.9 \\ 
\multicolumn{1}{l}{200} & \multicolumn{1}{l}{\multirow{3}[2]{*}{RMSE}} & 0.51
& 0.86 & 1.40 & 0.37 & 0.45 & 1.03 \\ 
\multicolumn{1}{l}{500} & \multicolumn{1}{l}{} & 0.15 & 0.20 & 0.17 & 0.19 & 
0.21 & 0.29 \\ 
\multicolumn{1}{l}{1000} & \multicolumn{1}{l}{} & 0.10 & 0.15 & 0.10 & 0.12
& 0.14 & 0.18 \\ 
\multicolumn{1}{l}{200} & \multicolumn{1}{l}{\multirow{3}[2]{4cm}{Coverage
probability (at 95\% level), \%}} & 95.3 & 88.4 & 93.4 & 90.0 & 92.0 & 84.7
\\ 
\multicolumn{1}{l}{500} & \multicolumn{1}{l}{} & 95.1 & 90.5 & 94.4 & 92.9 & 
93.1 & 88.9 \\ 
\multicolumn{1}{l}{1000} & \multicolumn{1}{l}{} & 95.3 & 92.1 & 95.1 & 94.8
& 93.8 & 91.7 \\ 
\multicolumn{1}{l}{200} & \multicolumn{1}{l}{\multirow{3}[2]{4cm}{Bias of
standard error estimates, \%}} & 19.3 & 16.0 & 39.5 & 16.1 & 22.8 & 12.7 \\ 
\multicolumn{1}{l}{500} & \multicolumn{1}{l}{} & 2.5 & 4.2 & 11.2 & 6.9 & 8.1
& 17.3 \\ 
\multicolumn{1}{l}{1000} & \multicolumn{1}{l}{} & 1.6 & 3.4 & 4.8 & 3.6 & 3.4
& 5.1 \\ 
\multicolumn{8}{c}{Probabilities} \\ 
\multicolumn{1}{l}{200} & \multicolumn{1}{l}{\multirow{3}[1]{*}{Bias, \%}} & 
2.3 & 1.5 & 4.4 & 5.1 & 3.3 & 3.1 \\ 
\multicolumn{1}{l}{500} & \multicolumn{1}{l}{} & 1.1 & 0.9 & 2.3 & 3.0 & 1.6
& 1.5 \\ 
\multicolumn{1}{l}{1000} & \multicolumn{1}{l}{} & 0.4 & 0.4 & 1.3 & 1.7 & 0.8
& 1.0 \\ 
\multicolumn{1}{l}{200} & \multicolumn{1}{l}{\multirow{3}[2]{*}{RMSE}} & 0.02
& 0.03 & 0.03 & 0.03 & 0.03 & 0.03 \\ 
\multicolumn{1}{l}{500} & \multicolumn{1}{l}{} & 0.01 & 0.02 & 0.02 & 0.02 & 
0.02 & 0.02 \\ 
\multicolumn{1}{l}{1000} & \multicolumn{1}{l}{} & 0.01 & 0.01 & 0.01 & 0.01
& 0.01 & 0.01 \\ 
\multicolumn{1}{l}{200} & \multicolumn{1}{l}{\multirow{3}[2]{4cm}{Coverage
probability (at 95\% level), \%}} & 94.4 & 94.4 & 95.3 & 95.3 & 95.1 & 94.8
\\ 
\multicolumn{1}{l}{500} & \multicolumn{1}{l}{} & 95.4 & 95.2 & 95.6 & 95.6 & 
95.9 & 95.7 \\ 
\multicolumn{1}{l}{1000} & \multicolumn{1}{l}{} & 95.5 & 95.5 & 95.7 & 95.7
& 95.6 & 95.6 \\ 
\multicolumn{1}{l}{200} & \multicolumn{1}{l}{\multirow{3}[2]{4cm}{Bias of
standard error estimates, \%}} & 4.2 & 4.2 & 6.8 & 6.4 & 5.5 & 15.1 \\ 
\multicolumn{1}{l}{500} & \multicolumn{1}{l}{} & 3.9 & 4.6 & 6.9 & 6.1 & 5.3
& 16.6 \\ 
\multicolumn{1}{l}{1000} & \multicolumn{1}{l}{} & 2.6 & 3.4 & 5.7 & 5.9 & 3.7
& 13.9 \\ 
\multicolumn{8}{c}{Marginal effects on probabilities} \\ 
\multicolumn{1}{l}{200} & \multicolumn{1}{l}{\multirow{3}[1]{*}{Bias, \%}} & 
4.5 & 4.1 & 10.6 & 16.9 & 11.5 & 23.0 \\ 
\multicolumn{1}{l}{500} & \multicolumn{1}{l}{} & 1.7 & 2.2 & 4.9 & 7.2 & 5.5
& 9.7 \\ 
\multicolumn{1}{l}{1000} & \multicolumn{1}{l}{} & 0.8 & 1.3 & 2.5 & 3.7 & 2.6
& 5.3 \\ 
\multicolumn{1}{l}{200} & \multicolumn{1}{l}{\multirow{3}[2]{*}{RMSE}} & 0.02
& 0.02 & 0.02 & 0.04 & 0.03 & 0.03 \\ 
\multicolumn{1}{l}{500} & \multicolumn{1}{l}{} & 0.01 & 0.01 & 0.01 & 0.02 & 
0.02 & 0.02 \\ 
\multicolumn{1}{l}{1000} & \multicolumn{1}{l}{} & 0.01 & 0.01 & 0.01 & 0.02
& 0.01 & 0.01 \\ 
\multicolumn{1}{l}{200} & \multicolumn{1}{l}{\multirow{3}[2]{4cm}{Coverage
probability (at 95\% level), \%}} & 89.4 & 87.7 & 91.7 & 87.9 & 94.6 & 91.8
\\ 
\multicolumn{1}{l}{500} & \multicolumn{1}{l}{} & 89.5 & 88.3 & 94.8 & 91.5 & 
95.0 & 93.0 \\ 
\multicolumn{1}{l}{1000} & \multicolumn{1}{l}{} & 89.3 & 88.6 & 95.3 & 93.9
& 95.1 & 93.9 \\ 
\multicolumn{1}{l}{200} & \multicolumn{1}{l}{\multirow{3}[2]{4cm}{Bias of
standard error estimates, \%}} & 4.7 & 5.7 & 8.1 & 6.1 & 21.4 & 39.1 \\ 
\multicolumn{1}{l}{500} & \multicolumn{1}{l}{} & 4.0 & 5.0 & 5.8 & 6.0 & 28.2
& 8.1 \\ 
\multicolumn{1}{l}{1000} & \multicolumn{1}{l}{} & 2.4 & 3.4 & 4.2 & 5.7 & 
12.7 & 7.4 \\ 
\bottomrule \bottomrule &  &  &  &  &  &  & 
\end{tabular}%
\end{table}

\end{document}
