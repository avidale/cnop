
\documentclass[letterpaper,fleqn,12pt]{article}
%%%%%%%%%%%%%%%%%%%%%%%%%%%%%%%%%%%%%%%%%%%%%%%%%%%%%%%%%%%%%%%%%%%%%%%%%%%%%%%%%%%%%%%%%%%%%%%%%%%%%%%%%%%%%%%%%%%%%%%%%%%%%%%%%%%%%%%%%%%%%%%%%%%%%%%%%%%%%%%%%%%%%%%%%%%%%%%%%%%%%%%%%%%%%%%%%%%%%%%%%%%%%%%%%%%%%%%%%%%%%%%%%%%%%%%%%%%%%%%%%%%%%%%%%%%%
\usepackage{geometry}
\usepackage[singlespacing]{setspace}
\usepackage{amsfonts}
\usepackage{inputenc}
\usepackage{graphicx}
\usepackage{amsmath}
\usepackage{accents}
\usepackage{eurosym}
\usepackage{amssymb}
\usepackage{rotating}
\usepackage{sectsty}
\usepackage{endnotes}
\usepackage{chbibref}
\usepackage{float}
\usepackage{nopageno}
\usepackage[labelsep=period]{caption}
\usepackage{scalefnt}
\usepackage{ragged2e}

\setcounter{MaxMatrixCols}{10}
%TCIDATA{OutputFilter=LATEX.DLL}
%TCIDATA{Version=5.50.0.2960}
%TCIDATA{<META NAME="SaveForMode" CONTENT="2">}
%TCIDATA{BibliographyScheme=Manual}
%TCIDATA{Created=Wednesday, April 07, 2010 09:52:31}
%TCIDATA{LastRevised=Wednesday, March 28, 2018 21:58:05}
%TCIDATA{<META NAME="GraphicsSave" CONTENT="32">}
%TCIDATA{<META NAME="DocumentShell" CONTENT="Standard LaTeX\Blank - Standard LaTeX Article">}
%TCIDATA{Language=American English}
%TCIDATA{CSTFile=40 LaTeX article.cst}

\newtheorem{theorem}{Theorem}
\newtheorem{acknowledgement}[theorem]{Acknowledgement}
\newtheorem{algorithm}[theorem]{Algorithm}
\newtheorem{axiom}[theorem]{Axiom}
\newtheorem{case}[theorem]{Case}
\newtheorem{claim}[theorem]{Claim}
\newtheorem{conclusion}[theorem]{Conclusion}
\newtheorem{condition}[theorem]{Condition}
\newtheorem{conjecture}[theorem]{Conjecture}
\newtheorem{corollary}[theorem]{Corollary}
\newtheorem{criterion}[theorem]{Criterion}
\newtheorem{definition}[theorem]{Definition}
\newtheorem{example}[theorem]{Example}
\newtheorem{exercise}[theorem]{Exercise}
\newtheorem{lemma}[theorem]{Lemma}
\newtheorem{notation}[theorem]{Notation}
\newtheorem{problem}[theorem]{Problem}
\newtheorem{proposition}[theorem]{Proposition}
\newtheorem{remark}[theorem]{Remark}
\newtheorem{solution}[theorem]{Solution}
\newtheorem{summary}[theorem]{Summary}
\newenvironment{proof}[1][Proof]{\noindent\textbf{#1.} }{\ \rule{0.5em}{0.5em}}
\geometry{left=1 in,right=1 in,top=1 in,bottom=1 in}
\input{tcilatex}
\setlength{\abovecaptionskip}{4pt}
\makeatletter
\def\@biblabel#1{\hspace*{-\labelsep}}
\newdimen\@bibhang \@bibhang=2em
\def\setbibhang#1{\@bibhang=#1}
\renewenvironment{thebibliography}[1]{  \setlength{\labelwidth}{0pt}
  \setlength{\labelsep}{0pt}
  \section*{\refname
     \@mkboth{\uppercase{\refname}}{\uppercase{\refname}}}   \vspace{0em}
   \setlength{\parindent}{0pt}
   \def\newblock{}
   \renewcommand{\bibitem}[2][]{     \if@filesw
       {\let\protect\noexpand\immediate
        \write\@auxout{\string\bibcite{##2}{##1}}}
        \fi\hangindent=\@bibhang\hangafter=1}}
\makeatother

\begin{document}

\title{Estimation of nested and cross-nested zero-inflated ordered probit
models\\
\bigskip }
\date{}
\author{David Dale\thanks{%
We gratefully acknowledge support from the Basic Research Program of the
Higher School of Economics.} \thanks{%
Corresponding author. Email:} \\
%EndAName
Yandex, Moscow \and Andrei Sirchenko \\
%EndAName
Higher School of Economics, Moscow}
\maketitle

\begin{abstract}
We develop the maximum likelihood estimators and provide the STATA commands, 
\texttt{nop}, \texttt{ziop-2} and \texttt{ziop-3}, which estimate the
three-part nested ordered probit model, the two-part cross-nested
zero-inflated ordered probit models of Harris and Zhao (2007, \textit{%
Journal of Econometrics }141(2): 1073--1099) and Brooks, Harris and Spencer
(2012,\ \textit{Economics Letters} 117(3): 683--686), and the three-part
cross-nested zero-inflated ordered probit model, with both exogenous and
endogenous switching. The zero-inflated models address the inflation of
neutral (zero) observations and allows zeros to emerge in two or three
latent regimes. The three-part models allow the probabilities of positive
and negative outcomes to be generated by distinct processes. We investigate
the finite-sample performance of proposed estimators by Monte Carlo
simulations, and illustrate the models with an empirical application to
federal funds rate target.

\bigskip \bigskip \bigskip \bigskip \bigskip \bigskip \bigskip 

\bigskip \bigskip \bigskip \bigskip \bigskip \bigskip \bigskip 

\bigskip \bigskip \bigskip \bigskip \bigskip \bigskip \bigskip 

\textbf{Keywords:} ordinal responses, nested ordered probit, zero inflation,
endogenous switching, Monte Carlo simulation, federal funds rate target.

\bigskip \bigskip \bigskip \bigskip \bigskip
\end{abstract}

%TCIMACRO{\TeXButton{Onehalf}{\begin{onehalfspace}}}%
%BeginExpansion
\begin{onehalfspace}%
%EndExpansion

\section{Introduction}

We introduce the STATA\ commands, \texttt{nop}, \texttt{ziop-2} and \texttt{%
ziop-3}, which estimate the two-level nested and cross-nested ordered-probit
models including the zero- and middle-inflated models of Harris and Zhao
(2007), Bagozzi and Mukherjee (2012) and Brooks, Harris and Spencer (2012).
The rationale behind the two-level nested decision process is standard in
the discrete-choice modeling when the set of alternatives faced by a
decision-maker can be partitioned into subsets (or nests) with similar
correlated alternatives. A choice among the nests and a choice among the
alternatives within each nest can be driven by different sets of observed
and unobserved factors (and common factors can have different weights). In
the case of unordered categorical data, in which choices can be grouped into
the nests of similar options, the nested logit model is a popular method.

The nested models for ordinal data are far more rare although the rationale
behind such approach is also straigtforward: choosing among a decrease, no
change or an increase is quite different from choosing between a small
decrease or a large decrease; and choosing between a small decrease\ or a
large decrease is quite different from choosing between a small increase\ or
a large increase. This leads to three implicit decisions: a regime decision
--- a choice among the nests, and two outcome decisions --- the choices of
the magnitude of decreases and increases (see the top left panel of Figure %
\ref{trees}).

\medskip 

%TCIMACRO{%
%\TeXButton{B}{\begin{figure}[H] \captionsetup{singlelinecheck = false, justification=justified}}}%
%BeginExpansion
\begin{figure}[H] \captionsetup{singlelinecheck = false, justification=justified}%
%EndExpansion
\caption{Nested and cross-nested zero-inflated ordered probit
models\label{trees}}%
%TCIMACRO{\TeXButton{center}{\centering}}%
%BeginExpansion
\centering%
%EndExpansion

\begin{center}
TBA
\end{center}

%TCIMACRO{\TeXButton{E}{\end{figure}}}%
%BeginExpansion
\end{figure}%
%EndExpansion

\medskip 

Furthermore, it would be useful for the no-change (zero) alternative to be
in three nests: its own one, another with decreases and another with
increases; so some no-change decisions can be driven by similar factors as
increases or decreases. This leads to a three-part cross-nested model with
the nests overlapping at a zero response; hence, the probability of zeros is
`inflated'. Since the regime decision is not observable, the zeros are
observationally equivalent --- it is never known to which of the three nests
the observed zero belongs. While several types of models with overlapping
nests for unordered categorical responses are developed (Vovsha 1997; Wen
and Koppelman 2001), the cross-nested models for ordinal\ outcomes are very
scarce.\footnote{%
Small (1987) introduced an ordered-choice model\ with overlapping nests,
which contain two adjacent choices.}

The prevalence of status quo, neutral or zero outcomes is observed in many
fields, including economics, sociology, technometrics, psychology and
biology. The heterogeneity of zeros is well recognized --- see Winkelmann
(2008) and Greene and Hensher (2010) for a review. Studies discriminate
among different types of zeros such as: no visits to doctor due to good
health, iatrophobia, or medical costs; no children due to infertility or
choice; no illness due to strong resistance or lack of infection. In the
studies of survey responses using an odd-point Likert-type scale, where the
respondents must indicate the negative, neutral or positive attitude or
opinion, the heterogeneity of indifferent responses (a true neutral option
versus an undecided, or ambivalent, or uninformed one, commonly reported as
neutral) is also well-recognized and sometimes labeled as the middle
category endorsement or inflation (Bagozzi and Mukhetjee 2012, Hern\'{a}%
ndez, Drasgow and Gonz\'{a}les-Rom\'{a} 2004, Kulas and Stachowski 2009). In
the decision-making experiments and micro-level studies of consumer choices,
election votes and other repeated choices, the prevalence of no-change
decisions is often attributed to the status quo bias -- a tendency to do
nothing or maintain one's previous decision, even though it is not always
objectively superior to the available options (Hartman, Doane and Woo 1991,
Kahneman, Knetsch, and Thaler 1991). It is a cognitive bias, explained by
both the rational causes (informational or cognitive limitations, transition
or analysis costs) and irrational ones (mental illusions and various
psychological inclinations such as convenience, habit, inertia, fear, innate
conservatism, loss aversion, and reputation concern) --- see Samuelson and
Zeckhauser (1988) for an excellent exposition.

The two-part zero-inflated models, developed to address the unobserved
heterogeneity of zeros, combines a binary choice model for the probability
of crossing the hurdle (to participate or not to participate; to consume or
not to consume) with a count or ordered-choice model for nonnegative
outcomes above the hurdle: the two parts are jointly estimated and the zero
observations can emerge in both parts. The two-part zero-inflated models
include the zero-inflated Poisson (Lambert 1992), negative binomial (Greene
1994), binomial (Hall 2002) and generalized Poisson (Famoye and Singh 2003)
models for count outcomes, and the zero-inflated ordered probit model
(Harris and Zhao 2007) and zero-inflated proportional odds model (Kelley and
Anderson 2008) for non-negative ordinal responses.

The model of Harris and Zhao (2007) is suitable for explaining decisions
such as the levels of consumption, when the upper hurdle is naturally binary
(to smoke or not to smoke) and the ordinal responses are typically
non-negative (see the top left panel of Figure \ref{trees}). Thus, the
abundant\ zeros are situated at one end of the ordered scale. Bagozzi and
Mukherjee (2012) and Brooks, Harris and Spencer (2012) extended the model of
Harris and Zhao (2007) and proposed the middle-inflated ordered probit model
for an ordinal outcome, which ranges from negative to positive responses,
and where an abundant outcome is situated in the middle of the choice
spectrum (see the bottom left panel of Figure \ref{trees}). The three-part
cross-nested zero-inflated ordered probit model (see the bottom right panel
of Figure \ref{trees}), introduced in Sirchenko (2013), is a natural
generalization of the models of Harris and Zhao (2007), Bagozzi and
Mukherjee (2012) and Brooks, Harris and Spencer (2012). A trichotomous
regime decision is more realistic and flexible than a binary decision
(change or no change) if applied to ordinal data with negative, zero and
positive values.

\section{\noindent Models}

The observed dependent variable $y_{t}$, $t=1,2,...,T$ is assumed to take on
a finite number of ordinal values $j$ coded as $%
\{-J^{-},...,-1,0,1,...,J^{+}\},$ where a typically predominant (and
potentially heterogeneous) response is coded as zero. The latent unobserved
(or only partially observed) variables are denoted by $\ast $. Each model
assumes an ordered-choice regime decision and the ordered-choice outcome
decisions conditional on each regime. The regime decision is allowed to be
correlated with each outcome decision. Each decision is modeled by an
ordered probit approach. We denote by $\mathbf{x}_{t},\mathbf{x}_{t}^{-},%
\mathbf{x}_{t}^{+}$ and $\mathbf{z}_{t}$ the $t^{\text{th}}$ rows of the
observed data matrices (which in addition to predetermined explanatory
variables may also include the lags of $y_{t}$), by $\mathbf{\beta ,\beta }%
^{-},\mathbf{\beta }$ and $\mathbf{\gamma }$ the vectors of unknown slope
parameters, by $\mathbf{\alpha ,\alpha }^{-}\mathbf{,\alpha }^{+}$ and $%
\mathbf{\mu }$ the vectors of unknown threshold parameters, by $\varepsilon
_{t},\varepsilon _{t}^{-},\varepsilon _{t}^{+}$ and $\nu _{t}\ $the error
terms that are independently and identically distributed (\textit{iid})
across $t$ with normal cumulative distribution function (CDF) $\Phi $ and
with variances $\sigma ^{2},\sigma _{-}^{2}.\sigma _{+}^{2},$ and $\sigma
_{\nu }^{2}$, respectively, and by $\Phi _{2}(g_{1}\mathbf{;}g_{2}\mathbf{;}%
\lambda )$ the CDF of the bivariate normal distribution of the two random
variables $g_{1}$ and\textbf{\ }$g_{2}$ with the correlation coefficient $%
\lambda $ and variances $\sigma _{1}^{2}$ and $\sigma _{2}^{2}$:

\begin{center}
$\Phi _{2}(g_{1}\mathbf{;}g_{2}\mathbf{;}\lambda )=\frac{1}{2\pi \sigma
_{1}\sigma _{2}\sqrt{1-\lambda ^{2}}}\underset{}{\underset{-\infty }{\overset%
{g_{1}}{\int }}}\underset{-\infty }{\overset{g_{2}}{\int }}\exp \left( -%
\frac{u^{2}/\sigma _{1}^{2}-2\lambda uw/\sigma _{1}\sigma _{2}+w^{2}/\sigma
_{2}^{2}}{2(1-\lambda ^{2})}\right) dudw.$
\end{center}

\subsubsection*{Three-part nested ordered probit (NOP) model}

Despite the wide-spread use of nested logit models for unordered categorical
responses we are not aware of any example of the nested ordered probit/logit
model for ordinal responses in the literature. The NOP model can be
described as

\medskip

$%
\begin{tabular}{ll}
\ Regime decision: & $r_{t}^{\ast }=\mathbf{z}_{t}\mathbf{\gamma }+\nu _{t},$
\ \ $s_{t}=\left\{ 
\begin{array}{rcl}
1 & \text{if} & \mu _{2}<r_{t}^{\ast }, \\ 
0 & \text{if} & \mu _{1}<r_{t}^{\ast }\leq \mu _{2}, \\ 
-1 & \text{if} & \text{ \ \ \ \ \ \ }r_{t}^{\ast }\leq \mu _{1}.%
\end{array}%
\right. $ \\ 
\ Outcome decisions: & $y_{t}^{-\ast }=\mathbf{x}_{t}^{-}\mathbf{\beta }%
^{-}+\varepsilon _{t}^{-},$ \ \ $y_{t}^{+\ast }=\mathbf{x}_{t}^{+}\mathbf{%
\beta }^{+}+\varepsilon _{t}^{+},$ \\ 
& $y_{t}=\left\{ 
\begin{array}{lcl}
j(j>0) & \text{if} & s_{t}=1\text{ and }\alpha _{j-1}^{+}<y_{t}^{+\ast }\leq
\alpha _{j}^{+}, \\ 
0 & \text{if} & s_{t}=0, \\ 
j(j<0) & \text{if} & s_{t}=-1\text{ \ \ and }\alpha _{j}^{-}<y_{t}^{-\ast
}\leq \alpha _{j+1}^{-},%
\end{array}%
\right. $ \\ 
& where $-\infty =\alpha _{0}^{+}\leq \alpha _{1}^{+}\leq ...\leq \alpha
_{J^{+}}^{+}=\infty $ \\ 
& and $-\infty =\alpha _{-J^{-}}^{-}\leq \alpha _{-J+1}^{-}\leq ...\leq
\alpha _{0}^{-}=\infty $. \\ 
\begin{tabular}{l}
Correlation among \\ 
decisions:%
\end{tabular}
& $\left[ 
\begin{array}{c}
\nu _{t} \\ 
\varepsilon _{t}^{i}%
\end{array}%
\right] \overset{iid}{\sim }\mathcal{N}\left( 
\begin{array}{c}
0 \\ 
0%
\end{array}%
,\left[ 
\begin{array}{cc}
\sigma _{\nu }^{2} & \rho _{i}\sigma _{\nu }\sigma _{i} \\ 
\rho _{i}\sigma _{\nu }\sigma _{i} & \sigma _{i}^{2}%
\end{array}%
\right] \right) $, $i\in \{-,+\}.$%
\end{tabular}%
$

\bigskip

The probabilities of the outcome $j$ in the NOP model are given by%
\begin{equation}
\begin{array}{l}
\Pr (y_{t}=j|\mathbf{z}_{t},\mathbf{x}_{t}^{-},\mathbf{x}_{t}^{+})=I_{j<0}%
\Pr (r_{t}^{\ast }\leq \mu _{1}\ \text{and }\alpha _{j}^{-}<y_{t}^{-\ast
}\leq \alpha _{j+1}^{-}|\mathbf{z}_{t},\mathbf{x}_{t}^{-}) \\ 
+I_{j=0}\Pr (\mu _{1}<r_{t}^{\ast }\leq \mu _{2}|\mathbf{z}_{t})+I_{j>0}\Pr
(\mu _{2}<r_{t}^{\ast }\ \text{and }\alpha _{j-1}^{+}<y_{t}^{+\ast }\leq
\alpha _{j}^{+}|\mathbf{z}_{t},\mathbf{x}_{t}^{+}) \\ 
=I_{j<0}\Pr (\nu _{t}\leq \mu _{1}-\mathbf{z}_{t}\mathbf{\gamma }\ \text{and 
}\alpha _{j}^{-}-\mathbf{x}_{t}^{-}\mathbf{\beta }^{-}<\varepsilon
_{t}^{-}\leq \alpha _{j+1}^{-}-\mathbf{x}_{t}^{-}\mathbf{\beta }^{-}) \\ 
+I_{j=0}\Pr (\mu _{1}-\mathbf{z}_{t}\mathbf{\gamma }<\nu _{t}\leq \mu _{2}-%
\mathbf{z}_{t}\mathbf{\gamma }) \\ 
+I_{j>0}\Pr (\mu _{2}-\mathbf{z}_{t}\mathbf{\gamma }<\nu _{t}\ \text{and }%
\alpha _{j-1}^{+}-\mathbf{x}_{t}^{+}\mathbf{\beta }^{+}<\varepsilon
_{t}^{+}\leq \alpha _{j}^{+}-\mathbf{x}_{t}^{+}\mathbf{\beta }^{+}) \\ 
=I_{j<0}[\Phi _{2}(\mu _{1}-\mathbf{z}_{t}\mathbf{\gamma };\alpha _{j+1}^{-}-%
\mathbf{x}_{t}^{-}\mathbf{\beta }^{-}\mathbf{;}\rho _{-})-\Phi _{2}(\mu _{1}-%
\mathbf{z}_{t}\mathbf{\gamma };\alpha _{j}^{-}-\mathbf{x}_{t}^{-}\mathbf{%
\beta }^{-}\mathbf{;}\rho _{-})] \\ 
+I_{j=0}[\Phi (\mu _{2}-\mathbf{z}_{t}\mathbf{\gamma })-\Phi (\mu _{1}-%
\mathbf{z}_{t}\mathbf{\gamma })] \\ 
+I_{j>0}[\Phi _{2}(-\mu _{2}+\mathbf{z}_{t}\mathbf{\gamma };\alpha _{j}^{+}-%
\mathbf{x}_{t}^{+}\mathbf{\beta }^{+};\mathbf{-}\rho _{+})-\Phi _{2}(-\mu
_{2}+\mathbf{z}_{t}\mathbf{\gamma };\alpha _{j-1}^{+}-\mathbf{x}_{t}^{+}%
\mathbf{\beta }^{+};\mathbf{-}\rho _{+})]\text{,}%
\end{array}
\label{Prob NOP}
\end{equation}

\noindent where $I_{j\leq 0}$ is an indicator function such that $I_{j\leq
0}=1$ if $j\leq 0$, and $I_{j\leq 0}=0$ if $j>0$ (analogously for $I_{j=0}$
and $I_{j\leq 0}$).

In the case of exogenous switching (when $\rho _{-}=\rho _{+}=0$), the
probabilities of the outcome $j$ in the NOP can be computed as

\begin{center}
$%
\begin{array}{l}
\Pr (y_{t}=j|\mathbf{z}_{t},\mathbf{x}_{t}^{-},\mathbf{x}_{t}^{+},\rho
_{-}=\rho _{+}=0) \\ 
=I_{j<0}\Phi (\mu _{1}-\mathbf{z}_{t}\mathbf{\gamma )}[\Phi (\alpha
_{j+1}^{-}-\mathbf{x}_{t}^{-}\mathbf{\beta }^{-})-\Phi (\alpha _{j}^{-}-%
\mathbf{x}_{t}^{-}\mathbf{\beta }^{-})] \\ 
+I_{j=0}[\Phi (\mu _{2}-\mathbf{z}_{t}\mathbf{\gamma })-\Phi (\mu _{1}-%
\mathbf{z}_{t}\mathbf{\gamma })] \\ 
+I_{j>0}[1-\Phi (\mu _{2}-\mathbf{z}_{t}\mathbf{\gamma })][\Phi (\alpha
_{j}^{+}-\mathbf{x}_{t}^{+}\mathbf{\beta }^{+})-\Phi (\alpha _{j-1}^{+}-%
\mathbf{x}_{t}^{+}\mathbf{\beta }^{+})]\text{.}%
\end{array}%
$
\end{center}

In the case of three outcome categories the NOP model degenerates to the
conventional ordered probit model.

\subsubsection*{Two-part cross-nested zero-inflated ordered probit (ZIOP-2)
model}

The ZIOP-2 model, which represents the middle-inflated ordered probit model
of Bagozzi and Mukherjee (2012) and Brooks, Harris and Spencer (2012), can
be described by the following system

\medskip

$%
\begin{tabular}{ll}
\ Regime decision: & $r_{t}^{\ast }=\mathbf{z}_{t}\mathbf{\gamma }+\nu _{t},$
\ \ $s_{t}^{\ast }=\left\{ 
\begin{array}{rcl}
1 & \text{if} & \mu <r_{t}^{\ast }, \\ 
0 & \text{if} & r_{t}^{\ast }\leq \mu .%
\end{array}%
\right. $ \\ 
\ Outcome decision: \ \ \ \ \ \  & $y_{t}^{\ast }=\mathbf{x}_{t}\mathbf{%
\beta }+\varepsilon _{t},$ \\ 
& $y_{t}=\left\{ 
\begin{array}{lcl}
j & \text{if} & s_{t}^{\ast }=1\text{ and }\alpha _{j-1}<y_{t}^{\ast }\leq
\alpha _{j}, \\ 
0 & \text{if} & s_{t}^{\ast }=0,%
\end{array}%
\right. $ \\ 
& where $-\infty =\alpha _{-J^{-}-1}\leq \alpha _{-J^{-}}\leq ...\leq \alpha
_{J^{+}}=\infty .$ \\ 
\begin{tabular}{l}
Correlation among \\ 
decisions:%
\end{tabular}
& $\left[ 
\begin{array}{c}
\nu _{t} \\ 
\varepsilon _{t}%
\end{array}%
\right] \overset{iid}{\sim }\mathcal{N}\left( 
\begin{array}{c}
0 \\ 
0%
\end{array}%
,\left[ 
\begin{array}{cc}
\sigma _{\nu }^{2} & \rho \sigma _{\nu }\sigma _{\varepsilon } \\ 
\rho \sigma _{\nu }\sigma _{\varepsilon } & \sigma _{\varepsilon }^{2}%
\end{array}%
\right] \right) .$%
\end{tabular}%
$

\bigskip

The probabilities of the outcome $j$ in the ZIOP-2 model are given by%
\begin{equation}
\begin{array}{l}
\Pr (y_{t}=j|\mathbf{z}_{t},\mathbf{x}_{t})=I_{j=0}\Pr (r_{t}^{\ast }\leq
\mu |\mathbf{z}_{t}) \\ 
+I_{j\geq 0}\Pr (\mu <r_{t}^{\ast }\ \text{and }\alpha _{j-1}<y_{t}^{\ast
}\leq \alpha _{j}|\mathbf{z}_{t},\mathbf{x}_{t}) \\ 
=I_{j=0}\Pr (\nu _{t}\leq \mu -\mathbf{z}_{t}\mathbf{\gamma })+I_{j\geq
0}\Pr (\mu -\mathbf{z}_{t}\mathbf{\gamma }<\nu _{t}\ \text{and }\alpha
_{j-1}-\mathbf{x}_{t}\mathbf{\beta }<\varepsilon _{t}\leq \alpha _{j}-%
\mathbf{x}_{t}\mathbf{\beta }) \\ 
=I_{j=0}\Phi (\mu -\mathbf{z}_{t}\mathbf{\gamma })+\Phi _{2}(-\mu +\mathbf{z}%
_{t}\mathbf{\gamma };\alpha _{j}-\mathbf{x}_{t}\mathbf{\beta };\mathbf{-}%
\rho )-\Phi _{2}(-\mu +\mathbf{z}_{t}\mathbf{\gamma };\alpha _{j-1}-\mathbf{x%
}_{t}\mathbf{\beta };\mathbf{-}\rho )\text{.}%
\end{array}
\label{Prob MIOP}
\end{equation}

In the case of exogenous switching (when $\rho =0$), the probabilities of
the outcome $j$ in the ZIOP-2 model can be computed as

\begin{center}
$%
\begin{array}{l}
\Pr (y_{t}=j|\mathbf{z}_{t},\mathbf{x}_{t},\rho =0)=I_{j=0}\Phi (\mu -%
\mathbf{z}_{t}\mathbf{\gamma }) \\ 
+[1-\Phi (\mu -\mathbf{z}_{t}\mathbf{\gamma })][\Phi (\alpha _{j}-\mathbf{x}%
_{t}\mathbf{\beta })-\Phi (\alpha _{j-1}-\mathbf{x}_{t}\mathbf{\beta })]%
\text{.}%
\end{array}%
$
\end{center}

If $y_{t}\geq 0$ for $\forall t,$ the ZIOP-2 model reduces to the model in
Harris and Zhao (2007).

\subsubsection*{Three-part cross-nested zero-inflated ordered probit
(ZIOP-3) model}

The ZIOP-3 model, developed by Sirchenko (2013), generalizes the ZIOP-2
model and can be described by the following system

\medskip

$%
\begin{tabular}{ll}
\ Regime decision: & $r_{t}^{\ast }=\mathbf{z}_{t}\mathbf{\gamma }+\nu _{t},$
\ \ $s_{t}^{\ast }=\left\{ 
\begin{array}{rcl}
1 & \text{if} & \mu _{2}<r_{t}^{\ast }, \\ 
0 & \text{if} & \mu _{1}<r_{t}^{\ast }\leq \mu _{2}, \\ 
-1 & \text{if} & \text{ \ \ \ \ \ \ }r_{t}^{\ast }\leq \mu _{1}.%
\end{array}%
\right. $ \\ 
\ Outcome decisions: & $y_{t}^{-\ast }=\mathbf{x}_{t}^{-}\mathbf{\beta }%
^{-}+\varepsilon _{t}^{-},$ \ \ $y_{t}^{+\ast }=\mathbf{x}_{t}^{+}\mathbf{%
\beta }^{+}+\varepsilon _{t}^{+},$ \\ 
& $y_{t}=\left\{ 
\begin{array}{lcl}
j(j\geq 0) & \text{if} & s_{t}^{\ast }=1\text{ \ \ and }\alpha
_{j-1}^{+}<y_{t}^{+\ast }\leq \alpha _{j}^{+}, \\ 
0 & \text{if} & s_{t}^{\ast }=0, \\ 
j(j\leq 0) & \text{if} & s_{t}^{\ast }=-1\text{ and }\alpha
_{j}^{-}<y_{t}^{-\ast }\leq \alpha _{j+1}^{-},%
\end{array}%
\right. $ \\ 
& where $-\infty =\alpha _{-1}^{+}\leq \alpha _{0}^{+}\leq ...\leq \alpha
_{J^{+}}^{+}=\infty $ \\ 
& and $-\infty =\alpha _{-J^{-}}^{-}\leq \alpha _{-J+1}^{-}\leq ...\leq
\alpha _{1}^{-}=\infty $. \\ 
\begin{tabular}{l}
Correlation among \\ 
decisions:%
\end{tabular}
& $\left[ 
\begin{array}{c}
\nu _{t} \\ 
\varepsilon _{t}^{i}%
\end{array}%
\right] \overset{iid}{\sim }\mathcal{N}\left( 
\begin{array}{c}
0 \\ 
0%
\end{array}%
,\left[ 
\begin{array}{cc}
\sigma _{\nu }^{2} & \rho _{i}\sigma _{\nu }\sigma _{i} \\ 
\rho _{i}\sigma _{\nu }\sigma _{i} & \sigma _{i}^{2}%
\end{array}%
\right] \right) $, $i\in \{-,+\}.$%
\end{tabular}%
$

\bigskip

The probabilities of the outcome $j$ in the ZIOP-3 model are given by

\begin{flushleft}
\begin{equation}
\begin{array}{l}
\Pr (y_{t}=j|\mathbf{z}_{t},\mathbf{x}_{t}^{-},\mathbf{x}_{t}^{+})=I_{j\leq
0}\Pr (r_{t}^{\ast }\leq \mu _{1}\ \text{and }\alpha _{j}^{-}<y_{t}^{-\ast
}\leq \alpha _{j+1}^{-}|\mathbf{z}_{t},\mathbf{x}_{t}^{-}) \\ 
+I_{j=0}\Pr (\mu _{1}<r_{t}^{\ast }\leq \mu _{2}|\mathbf{z}_{t})+I_{j\geq
0}\Pr (\mu _{2}<r_{t}^{\ast }\ \text{and }\alpha _{j-1}^{+}<y_{t}^{+\ast
}\leq \alpha _{j}^{+}|\mathbf{z}_{t},\mathbf{x}_{t}^{+}) \\ 
=I_{j\leq 0}\Pr (\nu _{t}\leq \mu _{1}-\mathbf{z}_{t}\mathbf{\gamma }\ \text{%
and }\alpha _{j}^{-}-\mathbf{x}_{t}^{-}\mathbf{\beta }^{-}<\varepsilon
_{t}^{-}\leq \alpha _{j+1}^{-}-\mathbf{x}_{t}^{-}\mathbf{\beta }^{-}) \\ 
+I_{j=0}\Pr (\mu _{1}-\mathbf{z}_{t}\mathbf{\gamma }<\nu _{t}\leq \mu _{2}-%
\mathbf{z}_{t}\mathbf{\gamma }) \\ 
+I_{j\geq 0}\Pr (\mu _{2}-\mathbf{z}_{t}\mathbf{\gamma }<\nu _{t}\ \text{and 
}\alpha _{j-1}^{+}-\mathbf{x}_{t}^{+}\mathbf{\beta }^{+}<\varepsilon
_{t}^{+}\leq \alpha _{j}^{+}-\mathbf{x}_{t}^{+}\mathbf{\beta }^{+}) \\ 
=I_{j\leq 0}[\Phi _{2}(\mu _{1}-\mathbf{z}_{t}\mathbf{\gamma };\alpha
_{j+1}^{-}-\mathbf{x}_{t}^{-}\mathbf{\beta }^{-}\mathbf{;}\rho _{-})-\Phi
_{2}(\mu _{1}-\mathbf{z}_{t}\mathbf{\gamma };\alpha _{j}^{-}-\mathbf{x}%
_{t}^{-}\mathbf{\beta }^{-}\mathbf{;}\rho _{-})] \\ 
+I_{j=0}[\Phi (\mu _{2}-\mathbf{z}_{t}\mathbf{\gamma })-\Phi (\mu _{1}-%
\mathbf{z}_{t}\mathbf{\gamma })] \\ 
+I_{j\geq 0}[\Phi _{2}(-\mu _{2}+\mathbf{z}_{t}\mathbf{\gamma };\alpha
_{j}^{+}-\mathbf{x}_{t}^{+}\mathbf{\beta }^{+};\mathbf{-}\rho _{+})-\Phi
_{2}(-\mu _{2}+\mathbf{z}_{t}\mathbf{\gamma };\alpha _{j-1}^{+}-\mathbf{x}%
_{t}^{+}\mathbf{\beta }^{+};\mathbf{-}\rho _{+})]\text{,}%
\end{array}
\label{Prob CroNOP}
\end{equation}
\end{flushleft}

\noindent where $I_{j\leq 0}$ is an indicator function such that $I_{j\leq
0}=1$ if $j\leq 0$, and $I_{j\leq 0}=0$ if $j>0$ (analogously for $I_{j=0}$
and $I_{j\leq 0}$).

In the case of exogenous switching (when $\rho _{-}=\rho _{+}=0$), the
probabilities of the outcome $j$ in the NOP can be computed as

\begin{center}
$%
\begin{array}{l}
\Pr (y_{t}=j|\mathbf{z}_{t},\mathbf{x}_{t}^{-},\mathbf{x}_{t}^{+},\rho
_{-}=\rho _{+}=0) \\ 
=I_{j\leq 0}\Phi (\mu _{1}-\mathbf{z}_{t}\mathbf{\gamma )}[\Phi (\alpha
_{j+1}^{-}-\mathbf{x}_{t}^{-}\mathbf{\beta }^{-})-\Phi (\alpha _{j}^{-}-%
\mathbf{x}_{t}^{-}\mathbf{\beta }^{-})] \\ 
+I_{j=0}[\Phi (\mu _{2}-\mathbf{z}_{t}\mathbf{\gamma })-\Phi (\mu _{1}-%
\mathbf{z}_{t}\mathbf{\gamma })] \\ 
+I_{j\geq 0}[1-\Phi (\mu _{2}-\mathbf{z}_{t}\mathbf{\gamma })][\Phi (\alpha
_{j}^{+}-\mathbf{x}_{t}^{+}\mathbf{\beta }^{+})-\Phi (\alpha _{j-1}^{+}-%
\mathbf{x}_{t}^{+}\mathbf{\beta }^{+})]\text{.}%
\end{array}%
$
\end{center}

The inflated outcome does not have to be in the \emph{very} middle of
ordered categories. If it is located at the \emph{end} of the ordered scale,
i.e. if $y_{t}\geq 0$ for $\forall t,$ the ZIOP-3 model reduces to the
ZIOP-2 model of Harris and Zhao (2007).

\subsubsection*{Maximum likelihood (ML) estimation}

The probabilities in each ordered probit model representing the regime and
outcome decisions can be consistently estimated under fairly general
conditions by an asymptotically normal ML estimator (Basu and de Jong 2007).
The simalteneous estimation of the ordered probit equations in the NOP,
ZIOP-2 and ZIOP-3 models can be performed using an ML estimator of the
vector of the parameters $\mathbf{\theta }$ that solves

\begin{equation}
\underset{\mathbf{\theta \epsilon \Theta }}{\max }\overset{}{\underset{}{%
\underset{t=1}{\overset{T}{\sum }}}}\overset{J^{+}}{\underset{j=-J^{-}}{\sum 
}}q_{tj}\ln [\Pr (y_{t}=j|\mathbf{x}_{t}^{all},\mathbf{\theta })]\text{,}
\label{LL}
\end{equation}

\noindent where $q_{tj}$ is an indicator function such that $q_{tj}=1$ if $%
y_{t}=j$ and $q_{itj}=0$ otherwise; $\mathbf{\theta }$ includes $\mathbf{%
\gamma ,}$ $\mathbf{\mu ,}$ $\mathbf{\beta }^{-},$ $\mathbf{\beta }^{+},$ $%
\mathbf{\alpha }^{-}$ and $\mathbf{\alpha }^{+}$ for the NOP model, $\mathbf{%
\gamma ,}$ $\mathbf{\mu ,}$ $\mathbf{\beta },$ $\mathbf{\alpha }$ and $%
\mathbf{\rho }$ for the ZIOP-2 model, and $\mathbf{\gamma ,}$ $\mathbf{\mu ,}
$ $\mathbf{\beta }^{-},$ $\mathbf{\beta }^{+},$ $\mathbf{\alpha }^{-},$ $%
\mathbf{\alpha }^{+}\mathbf{,}$ $\mathbf{\rho }^{-}$ and $\mathbf{\rho }^{+}$
for the ZIOP-3 model; $\mathbf{\Theta }$ is a parameters' space; $\mathbf{x}%
_{t}^{all}$ is a vector that contains the values of all unique covariates
from all equations of the model at observation $t$; and $\Pr (y_{t}=j|%
\mathbf{x}_{t}^{all},\mathbf{\theta })$ are probabilities in either (\ref%
{Prob NOP}), or (\ref{Prob CroNOP}), or (\ref{Prob MIOP}) for, respectively,
the NOP, ZIOP-2 and ZIOP-3 models.

The intercept components of $\mathbf{\beta ,}$ $\mathbf{\beta }^{-},$ $%
\mathbf{\beta }$ and $\mathbf{\gamma }$ are identified up to scale and
location, that is only jointly with the corresponding threshold parameters $%
\mathbf{\alpha ,}$ $\mathbf{\alpha }^{-}\mathbf{,}$ $\mathbf{\alpha }^{+}$
and $\mathbf{\mu }$ and variances $\sigma ^{2},$ $\sigma _{-}^{2},$ $\sigma
_{+}^{2},$ and $\sigma _{\nu }^{2}$. As is common in the identification of
discrete choice models, the variances $\sigma ^{2},$ $\sigma _{-}^{2},$ $%
\sigma _{+}^{2},$ and $\sigma _{\nu }^{2}$ are fixed to unit, and the
intercept components of $\mathbf{\beta ,}$ $\mathbf{\beta }^{-},$ $\mathbf{%
\beta }$ and $\mathbf{\gamma }$ are fixed to zero. The probabilities in (\ref%
{Prob NOP}), (\ref{Prob MIOP}) and (\ref{Prob CroNOP}) are invariant to
these (arbitrary) identifying assumptions. Up to scale and location, we can
identify all parameters in $\mathbf{\theta }$ because of the nonlinearity of
ordered probit equations, i.e. via the functional form (Heckman 1978; Wilde
2000). However, since the normal CDF is approximately linear in the middle
of its support, the simultaneous estimation of two or three equations may
experience a weak identification problem if all decision and outcome
equations contain the same set of covariates. To enhance the precision of
parameter estimates we may impose exclusion restrictions on the
specification of covariates in each equation. The asymptotic standard errors
of $\widehat{\mathbf{\theta }}$ can be computed from the Hessian matrix.

The three regimes (nests) in the NOP model are fully observable, contrary to
the latent (only partially observed) regimes in the ZIOP-2 and ZIOP-3
models. The log of the likelihood function of the NOP model ---\ again in
contrast with the ZIOP-2 and ZIOP-3 models --- is separable with respect to
the parameters in the three equations. Thus, solving (\ref{LL}) for the NOP
model is equivalent to maximizing separately the likelihoods of the three
ordered probit models, representing the regime and outcome decisions in (\ref%
{incl}), if the data matrices in the outcome decisions are truncated to
contain only those rows $\mathbf{x}_{t}^{-}$ or $\mathbf{x}_{t}^{-}$ for
which $y_{t}<0$ or $y_{t}>0$, respectively.

\subsubsection*{Marginal effects}

\noindent The marginal effect (ME) of a continuous covariate $k$ (the $k^{%
\text{th}}$ element of $\mathbf{x}_{t}^{all}$) on the probability of each
discrete outcome $j$ are computed for the ZIOP-3 model as

$%
\begin{array}{l}
ME_{k,j,t}=\frac{\partial \Pr (y_{t}=j|\mathbf{\theta })}{\partial \mathbf{x}%
_{t,k}^{all}}=I_{j\leq 0}\left\{ \left[ \Phi \left( \frac{\mu _{1}-\mathbf{z}%
_{t}\mathbf{\gamma }-\rho _{-}(\alpha _{j}^{-}-\mathbf{x}_{t}^{-}\mathbf{%
\beta ^{-})}}{\sqrt{1-(\rho _{-})^{2}}}\right) f(\alpha _{j}^{-}-\mathbf{x}%
_{t}^{-}\mathbf{\beta ^{-}})\right. \right. \\ 
\left. -\Phi \left( \frac{\mu _{1}-\mathbf{z}_{t}\mathbf{\gamma }-\rho
_{-}(\alpha _{j+1}^{-}-\mathbf{x}_{t}^{-}\mathbf{\beta ^{-})}}{\sqrt{1-(\rho
_{-})^{2}}}\right) f(\alpha _{j+1}^{-}-\mathbf{x}_{t}^{-}\mathbf{\beta ^{-}})%
\right] \mathbf{\beta }_{k}^{-all} \\ 
\left. -\left[ \Phi \left( \frac{\alpha _{j+1}^{-}-\mathbf{x}_{t}^{-}\mathbf{%
\beta ^{-}}-\rho _{-}(\mu _{1}-\mathbf{z}_{t}\mathbf{\gamma )}}{\sqrt{%
1-(\rho _{-})^{2}}}\right) -\Phi \left( \frac{\alpha _{j}^{-}-\mathbf{x}%
_{t}^{-}\mathbf{\beta ^{-}}-\rho _{-}(\mu _{1}-\mathbf{z}_{t}\mathbf{\gamma )%
}}{\sqrt{1-(\rho _{-})^{2}}}\right) \right] f(\mu _{1}-\mathbf{z}_{t}\mathbf{%
\gamma })\mathbf{\gamma }_{k}^{all}\right\} \\ 
-I_{j=0}[f(\mu _{2}-\mathbf{z}_{t}\mathbf{\gamma })-f(\mu _{1}-\mathbf{z}_{t}%
\mathbf{\gamma })]\mathbf{\gamma }_{k}^{all} \\ 
+I_{j\geq 0}\left\{ \left[ \Phi \left( \frac{\mathbf{z}_{t}\mathbf{\gamma }%
-\mu _{2}+\rho _{+}(\alpha _{j-1}^{+}-\mathbf{x}_{t}^{+}\mathbf{\beta ^{+})}%
}{\sqrt{1-(\rho _{+})^{2}}}\right) \right. \right. f(\alpha _{j-1}^{+}-%
\mathbf{x}_{t}^{+}\mathbf{\beta ^{+}}) \\ 
\left. -\Phi \left( \frac{\mathbf{z}_{t}\mathbf{\gamma }-\mu _{2}+\rho
^{+}(\alpha _{j}^{+}-\mathbf{x}_{t}^{+}\mathbf{\beta ^{+})}}{\sqrt{1-(\rho
_{+})^{2}}}\right) f(\alpha _{j}^{+}-\mathbf{x}_{t}^{+}\mathbf{\beta ^{+}})%
\right] \mathbf{\beta }_{k}^{+all} \\ 
\left. +\left[ \Phi \left( \frac{\alpha _{j}^{+}-\mathbf{x}_{t}^{+}\mathbf{%
\beta ^{+}}+\rho _{+}(\mathbf{z}_{t}\mathbf{\gamma }-\mu _{2}\mathbf{)}}{%
\sqrt{1-(\rho _{+})^{2}}}\right) -\Phi \left( \frac{\alpha _{j-1}^{+}-%
\mathbf{x}_{t}^{+}\mathbf{\beta ^{+}}+\rho _{+}(\mathbf{z}_{t}\mathbf{\gamma 
}-\mu _{2}\mathbf{)}}{\sqrt{1-(\rho _{+})^{2}}}\right) \right] f(\mathbf{z}%
_{t}\mathbf{\gamma }-\mu _{2})\mathbf{\gamma }_{k}^{all}\right\} ,%
\end{array}%
$

\medskip

\noindent where $f$ is the probability density function of the standard
normal distribution, and $\mathbf{\gamma }_{k}^{all}$, $\mathbf{\beta }%
_{k}^{-all}$ and $\mathbf{\beta }_{k}^{+all}$ are the parameters on the $k^{%
\text{th}}$ covariate in $\mathbf{x}_{t}^{all}$ in each equation ($\mathbf{%
\gamma }_{k}^{all}$, $\mathbf{\beta }_{k}^{-all}$ or $\mathbf{\beta }%
_{k}^{+all}$ is zero if the $k^{\text{th}}$ covariate in $\mathbf{x}%
_{t}^{all}$ is not included into the corresponding equation). For a
discrete-valued covariate, the ME can be computed as the change in the
probabilities when this covariate changes by one increment and all other
covariates are fixed.

The MEs for the NOP model are given by replacing $I_{j\geq 0}$ with $I_{j>0}$
and $I_{j\leq 0}$ with $I_{j<0}$. The MEs for the ZIOP-2 model are computed
as

\medskip

$%
\begin{array}{l}
ME_{k,j,t}=\frac{\partial \Pr (y_{t}=j|\mathbf{\theta })}{\partial \mathbf{x}%
_{t,k}^{all}}=-I_{j=0}[f(\mu -\mathbf{z}_{t}\mathbf{\gamma })]\mathbf{\gamma 
}_{k}^{all} \\ 
+\left[ \Phi \left( \frac{\mathbf{z}_{t}\mathbf{\gamma }-\mu +\rho (\alpha
_{j-1}-\mathbf{x}_{t}\mathbf{\beta )}}{\sqrt{1-\rho ^{2}}}\right) f(\alpha
_{j-1}-\mathbf{x}_{t}\mathbf{\beta })-\Phi \left( \frac{\mathbf{z}_{t}%
\mathbf{\gamma }-\mu +\rho (\alpha _{j}-\mathbf{x}_{t}\mathbf{\beta )}}{%
\sqrt{1-\rho ^{2}}}\right) f(\alpha _{j}-\mathbf{x}_{t}\mathbf{\beta })%
\right] \mathbf{\beta }_{k}^{all} \\ 
+\left[ \Phi \left( \frac{\alpha _{j}-\mathbf{x}_{t}\mathbf{\beta }+\rho (%
\mathbf{z}_{t}\mathbf{\gamma }-\mu \mathbf{)}}{\sqrt{1-\rho ^{2}}}\right)
-\Phi \left( \frac{\alpha _{j-1}-\mathbf{x}_{t}\mathbf{\beta }+\rho (\mathbf{%
z}_{t}\mathbf{\gamma }-\mu \mathbf{)}}{\sqrt{1-\rho ^{2}}}\right) \right] f(%
\mathbf{z}_{t}\mathbf{\gamma }-\mu )\mathbf{\gamma }_{k}^{all}\text{.}%
\end{array}%
$

\medskip

The asymptotic standard errors of the MEs are computed using the Delta
method as the square roots of the diagonal elements of

\begin{center}
$\widehat{Var(\widehat{\underset{}{\mathbf{ME}_{k,j,t}}})}=\nabla _{\theta }%
\widehat{\mathbf{ME}}_{k,j,t}\widehat{Var(\widehat{\mathbf{\theta }})}\nabla
_{\theta }\widehat{\mathbf{ME}}_{k,j,t})^{\prime }$.
\end{center}

\subsubsection*{\noindent The relations among the models and their comparison%
}

In this section I discuss the relations among the NOP, ZIOP-2 and ZIOP-3
models and the choice of a formal model-selection test, which depends on
whether the models are nested in each other.

The exogenous-switching version of each model is nested in the
endogenous-switching version of this model as its uncorrelated special case;
their comparison can be performed using any classical lilkelihood-based test
for nested hypotheses, such as the likelihood ratio (LR) test.

The NOP model is nested in the ZIOP-3 model. The latter becomes a NOP model
if $\alpha _{-1}^{-}\rightarrow \infty $ and $\alpha _{1}^{+}\rightarrow
-\infty $; therefore, $\Pr (y_{t}=0|\mathbf{x}_{t}^{+},s_{t}=1)\rightarrow 0$
and $\Pr (y_{t}=0|\mathbf{x}_{t}^{-},s_{t}=-1)\rightarrow 0$. Thus, the
comparison of the NOP and ZIOP-3 models can also be performed with the LR
test; however, the critical values of the classical LR test are invalid
since some standard regularity conditions of the classical LR test fail to
hold. In particular, the values of $\alpha _{-1}^{-}$ and $\alpha _{1}^{+}$
in the null hypothesis are not the interior points of the parameter space;
hence, the asymptotic distribution of the LR statistics is not standard.
Instead, we may use the simulated critical values provided in Andrews (2001).

Generally, the ZIOP-2 model is not a special case of the ZIOP-3 model, and
vice versa. However, they are not strictly non-nested and overlap if all
their slope parameters are fixed to zeros. We can compare them using a
likelihood-based test for non-nested overlapping models, such as the Vuong
test (Vuong 1989). A special case when the ZIOP-3 model nests the ZIOP-2
model emerges under some restrictions on the parameters provided: (i) $y_{t}$
only has three choices, (ii) the regressors in $\mathbf{x}_{t}^{-}$ and $%
\mathbf{x}_{t}^{+}$ in the outcome equations of the ZIOP-3 model contain all
regressors in $\mathbf{z}_{t}$ in the ZIOP-2 regime equation, and (iii) the
regressors in $\mathbf{z}_{t}$ in the regime equation of the ZIOP-3 model
include all regressors in the $\mathbf{x}_{t}$ in the ZIOP-2 amount equation
(see Appendix for the details). In this case, the selection between the
ZIOP-3 and ZIOP-2 models can be performed using any classical
likelihood-based test for nested hypotheses, which can be interpreted as a
misspecification test for the latter.

\section{The nop, ziop-2 and ziop-3 commands}

\subsection*{Syntax}

%\subsubsection{Syntax}

\hangindent=\parindent\noindent \texttt{ziop-3 $depvar$ $indepvars$ [$if$] [$%
in$] [, zp($varlist$) zn($varlist$) infcat($integer$ $0$) correlated cluster(%
$varname$) robust initial($string$)] }

This command estimates by ML the three-part cross-nested zero-inflated
ordered probit model with possibly different sets of covariates in the
regime and outcome equations and possibly endogenous switching among three
latent regimes.

\hangindent=\parindent\noindent \texttt{ziop-2 $depvar$ $indepvars$ [$if$] [$%
in$] [, z ($varlist$) infcat($integer$ $0$) correlated cluster($varname$)
robust initial($string$)] }

This command  estimates by ML the two-part cross-nested zero-inflated
ordered probit model with possibly different sets of covariates in the
regime and outcome equations and possibly endogenous switching among two
latent regimes.

\hangindent=\parindent
\noindent \texttt{nop $depvar$ $indepvars$ [$if$] [$in$] [, zp($varlist$) zn(%
$varlist$) infcat($integer$ $0$) correlated cluster($varname$) robust
initial($string$)] }

This command estimates by ML the three-part nested ordered probit model with
possibly different sets of covariates in the regime and outcome equations
and possibly endogenous switching among three latent regimes..

%\subsubsection{Description}

\subsubsection*{Options}

\begin{tabular}{lp{12cm}}
\textit{options} & Description \\ 
\midrule \texttt{zp($varlist$)} & list of covariates for positive response
in NOP and CNOP models; by default, it equals $indepvars$, the list of
covariates for initial stage \\ 
\texttt{zn($varlist$)} & list of covariates for negative response in NOP and
CNOP models; by default, it equals $indepvars$, the list of covariates for
initial stage \\ 
\texttt{z($varlist$)} & list of covariates for non-zero response in ZIOP
models; by default, it equals $indepvars$, the list of covariates for
initial stage \\ 
\texttt{infcat($integer$)} & value of the response variable that should be
modeled as inflated; by default, it equals 0 \\ 
\texttt{correlated} & flag that errors in the first and second stages may be
correlated, forcing estimation of CNOPc, NOPc or ZIOPc model \\ 
\texttt{robust} & flag that variance-covariance estimator must be robust
(based on ``sandwich``) estimate \\ 
\texttt{cluster($varname$)} & clustering variable for robust variance
estimator \\ 
\texttt{initial($string$)} & whitespace-delimited list of initial parameter
values for estimation, in the following order: $\beta$, $\alpha$, $%
\gamma^{+} $, $\mu^{+}$, $\gamma^{-}$, $\mu^{-}$, $\rho^{-}$, $\rho^{+}$%
\end{tabular}

\subsubsection*{Examples}

TBD

\subsubsection*{Stored results}

\texttt{nop}, \texttt{ziop-2}, and \texttt{ziop-3} store the following in 
\texttt{e()}:

%Scalars

\begin{tabular}{p{3cm}p{12cm}}
\texttt{e(N)} & number of observations%
\end{tabular}

%Macros

\begin{tabular}{p{3cm}p{12cm}}
\texttt{e(cmd)} & \texttt{cnop}, \texttt{nop}, or \texttt{miop}, respectively
\\ 
\texttt{e(depvar)} & dependent variable of regression%
\end{tabular}

%Matrices

\begin{tabular}{p{3cm}p{12cm}}
\texttt{e(b)} & parameters vector \\ 
\texttt{e(V)} & variance-covariance matrix%
\end{tabular}

%Functions

\begin{tabular}{p{3cm}p{12cm}}
\texttt{e(sample)} & marks estimation sample%
\end{tabular}

\subsection*{Postestimation commands}

\subsubsection*{The predict command}

The \texttt{predict} command after the nop, ziop-2 and ziop-3 estimation
commands produces either predicted probabilities or expected values of the
responses.

\texttt{predict $varname$ [$if$] [$in$] [, zeroes regime output($string$) at(%
$string$)]}

\texttt{name} is the name of predicted variable, if it is single, or prefix
for names, if there are several predicted variables

\texttt{zeroes} indicates that different types of zeroes (i.e. ``intrinsic
zeroes``, or ``positive zeroes``, or ``negative zeroes``) must be predicted
instead of different response values.

\texttt{regime} indicates that different groups of response (negative,
positive or zero) must be predicted instead of different response values.
This option is ignored if \texttt{zeroes} option is on.

\texttt{output(string)} specifies type of aggregating predicted
probabilities of different response. Possible values are \texttt{mode} and 
\texttt{mean}, for predicting average or most probable outcome, and \texttt{%
cum} for predicting cumulative response probabilities (i.e. $\Pr (y_{t}<=-2)$%
, $\Pr (y_{t}<=-1)$, $\Pr (y_{t}<=0)$ etc.). If not specified, raw response
probabilities are predicted ($\Pr (y_{t}=-2)$, $\Pr (y_{t}=-1)$, $\Pr
(y_{t}=0)$ etc.).

\subsubsection*{The cnopmargins command}

\texttt{cnopmargins [, at($string$) nominal($varlist$) zeroes regime]}

This command prints marginal effects for the last estimated model (either
NOP, or ZIOP-2, or ZIOP-3), calculated at the specified point, along with
confidence intervals.

\texttt{at(string)} specifies at which point predictions must be calculated.
If at is specified, (as a list of \texttt{varname=value} expressions,
separated by comma), prediction is calculated at this point and posted on
the screen without saving to the dataset. If some covariate names are not
specified, their mean value is taken instead.

\texttt{nominal} is a space-separated list of covariates which should be
considered as nominal; marginal effect is then calculated as difference
between values at 0 and at 1.

\texttt{zeroes} and \texttt{regime} indicate that marginal effects should be
calculated for different zeroes or for groups of response variable, as in 
\texttt{predict} command.

\subsubsection*{The cnopprobabilities command}

\texttt{cnopprobabilities [, at($string$) zeroes regime]}

This command prints predicted probabilities for the last estimated model
(either NOP, or ZIOP-2, or ZIOP-3) , calculated at the specified point,
along with confidence intervals. The point \texttt{at} is specified like in 
\texttt{cnopmargins}.

\subsubsection*{The cnopcontrasts command}

\texttt{cnopcontrasts [, at($string$) to($string$) zeroes regime] }

This command prints differences in predicted probabilities for the last
estimated model (either NOP, or ZIOP-2, or ZIOP-3), calculated between the
specified points, along with confidence intervals. The points \texttt{at}
and \texttt{to} are specified like \texttt{at} in \texttt{cnopmargins}.

\subsubsection*{Examples}

TBD

\section{Monte Carlo simulations}

We conducted extensive Monte Carlo experiments to illustrate the finite
sample performance of the ML estimators of each model.

\subsection*{Monte Carlo design}

We simulated six processes generated by the NOP, ZIOP-2 and ZIOP-3 models
with both exogenous and endogenous switching. The repeated samples with 200,
500 and 1000 observations were independently generated and then estimated by
the true model. The number of replications was 10,000 in each experiment.

Three covariates $\mathbf{w}_{\mathbf{1}}$, $\mathbf{w}_{\mathbf{2}}$ and $%
\mathbf{w}_{\mathbf{3}}$ were drawn in each replication as\noindent\ $%
\mathbf{w}_{\mathbf{1}}\overset{\emph{iid}}{\sim }\mathcal{N}(0,1)+2$, $%
\mathbf{w}_{\mathbf{2}}\overset{\emph{iid}}{\sim }\mathcal{N}(0,1$), and $%
\mathbf{w}_{\mathbf{3}}=-1$ if $\mathbf{u}\leq 0.3$, $0$ if $0.3<\mathbf{u}%
\leq 0.7$, or $1$ if $\mathbf{u}>0.7$, where $\mathbf{u}\overset{\emph{iid}}{%
\sim }\mathcal{U}[0,1]$. The repeated samples were generated for the NOP and
ZIOP-2 DGPs\textit{\ }with $\mathbf{Z=(w_{1}},\mathbf{w_{2}}$), $\mathbf{%
\mathbf{X}^{-}=(w_{1}},\mathbf{w_{3}}$), $\mathbf{\mathbf{X}^{+}=(w_{2}},%
\mathbf{w_{3}}$), and for the ZIOP-2 DGP with $\mathbf{Z=(w}_{1},\mathbf{w}%
_{3})$, $\mathbf{\mathbf{X}=(w}_{2},\mathbf{w}_{3})$. The dependent variable 
$y$ was generated with five outcome categories: -2, -1, 0, 1 and 2. The
parameters were calibrated to yield on average the following frequencies of
the above outcomes: 7\%, 14\%, 58\%, 14\% and 7\%, respectively. To avoid
the divergence of ML estimates due to the problem of complete separation
(perfect prediction), which could happen if actual number of observations in
any outcome category is very low, the samples with any outcome category
frequency lower than 6\% were re-generated. The matrix of the MEs has $%
3\times 5=15$ elements; their values, which depend on the values of the
regressors, are computed at the population medians of the covariates. The
variances of the errors in the regime and outcome equations were fixed to
one. The true values of all other parameters in the simulations are shown in
Table \ref{True param}. The simulations and estimations were performed using
the MATA programming language. The starting values for slope and threshold
parameters were obtained using the independent ordered-probit estimations of
each equation. The starting values for $\rho $, $\rho _{-}$ and $\rho _{+}$
were obtained by maximizing the logarithms of the likelihood functions of
the endogenous-switching models holding the other parameters fixed at their
estimates in the corresponding exogenous-switching model.

% Table generated by Excel2LaTeX from sheet 'params'

%TCIMACRO{%
%\TeXButton{B}{\begin{table}[H] \captionsetup{singlelinecheck = false, justification=justified}}}%
%BeginExpansion
\begin{table}[H] \captionsetup{singlelinecheck = false, justification=justified}%
%EndExpansion

\caption{Monte Carlo simulations: The true values of parameters\label{True
param}}%
%TCIMACRO{\TeXButton{TeX field}{\centering}}%
%BeginExpansion
\centering%
%EndExpansion

\begin{center}
TBA
\end{center}

%TCIMACRO{\TeXButton{E}{\end{table}}}%
%BeginExpansion
\end{table}%
%EndExpansion

\subsection*{Monte Carlo results}

Table \ref{MC_results} reports the measures of accuracy for the estimates of
the probabilities and MEs. For each model, the bias and RMSE decrease as
sample size increases. RMSE decreases in most cases faster than asymptotic
rate $\sqrt{n}$. This may be caused by a small number of large deviations in
parameter estimation in small samples. For most of models and sample sizes,
the bias and RMSE are slightly higher for the endogenous-switching version.
This is expected from a more complex model, estimated with the same sample
size.

Standard error estimates for parameters on average correspond to the actual
standard errors. Large deviations make standard errors estimates biased,
especially on small samples, but this problem rapidly decreases as sample
size grows. Anyway, rare large deviations do not prevent asymptotic coverage
probabilities of 95\% confidence intervals from being consistent. In
general, results of Monte Carlo simulations show that estimators of the
proposed nested and cross-nested ordered probit models are consistent, but
should be used carefully in small samples. As a rule of thumb, we would
advise using at least 10 observations per variable in each outcome class,
which corresponds to 1000 observations in our case (make it more
optimistic!!!).

\bigskip

%TCIMACRO{%
%\TeXButton{B}{\begin{table}[H] \captionsetup{singlelinecheck = false, justification=justified}}}%
%BeginExpansion
\begin{table}[H] \captionsetup{singlelinecheck = false, justification=justified}%
%EndExpansion

\caption{Monte Carlo results: The accuracy of ML
estimators\label{MC_results}}%
%TCIMACRO{\TeXButton{TeX field}{\centering}}%
%BeginExpansion
\centering%
%EndExpansion

\begin{center}
%TCIMACRO{%
%\FRAME{ihF}{6.2349in}{5.2453in}{0in}{}{}{Figure}{%
%\special{language "Scientific Word";type "GRAPHIC";maintain-aspect-ratio TRUE;display "USEDEF";valid_file "T";width 6.2349in;height 5.2453in;depth 0in;original-width 6.2075in;original-height 5.2154in;cropleft "0";croptop "1";cropright "1";cropbottom "0";tempfilename 'P6BDD601.wmf';tempfile-properties "XPR";}}}%
%BeginExpansion
\includegraphics[
natheight=5.2154in, natwidth=6.2075in, height=5.2453in, width=6.2349in]
{C:/Users/Andrei/Documents/Dale/Our paper/graphics/P6BDD601__1.pdf}%
%EndExpansion
\end{center}

%TCIMACRO{\TeXButton{TeX field}{\justify}}%
%BeginExpansion
\justify%
%EndExpansion
%TCIMACRO{\TeXButton{TeX field}{\footnotesize}}%
%BeginExpansion
\footnotesize%
%EndExpansion

Notes: (exog) -- exogenous switching; Bias -- the absolute difference
between the estimated and true values, devided by the true value; RMSE --
the absolute root mean square error of the estimates; Coverage probability
-- the percentage of times the estimated asymptotic 95\% confidence
intervals cover the true values; Bias of standard error estimates -- the
absolute difference between the average of the estimated asymptotic standard
errors of the estimates and the standard deviation of the estimates in all
replications. The above measures of accuracy for the estimates of the
probabilities are averaged across five outcome categories, and for the
estimates of the MEs are averaged across five outcome categories and across
all covariates.

%TCIMACRO{\TeXButton{E}{\end{table}}}%
%BeginExpansion
\end{table}%
%EndExpansion

% Need to insert \midrule between parts of table and 4cm width of multirows instead of * manually
% Need \usepackage{booktabs} and \usepackage{multirow}
% Table generated by Excel2LaTeX from sheet 'total'. 

\section{Empirical application}

The existing applications of discrete-choice approach to monetary policy
rules (e.g., Hamilton and Jorda, 2002; Hu and Phillips, 2004; Dolado et al.,
2005; Piazzesi, 2005; Basu and de Jong, 2007; Kauppi, 2012; Van den Hauwe et
al., 2013) do not allow for a regime-switching behavior of central bank.

TBA

\section*{\noindent \noindent Concluding remarks}

\noindent

\begin{thebibliography}{99}
\bibitem{} Andrews, D. W. K. 2001. Testing when a parameter is on the
boundary of the maintained hypothesis.\ \textit{Econometrica} 69 (3):
683--734.

\bibitem{} Bagozzi, B. E., and B. Mukherjee. 2012. A mixture model for
middle category inflation in ordered survey responses.\ \textit{Political
Analysis\ }20: 369--386.

\bibitem{} Basu, D., and R. M. de Jong. 2007. Dynamic multinomial ordered
choice with an application to the estimation of monetary policy rules.\ 
\textit{Studies in Nonlinear Dynamics and Econometrics} 11 (4): 1--35.

\bibitem{} Brooks, R., M. N. Harris, and C. Spencer. 2012. Inflated ordered
outcomes.\ \textit{Economics Letters} 117 (3): 683--686.

\bibitem{} Famoye, F., and K. P. Singh. 2003. On inflated generalized
Poisson regression models. \textit{Advanced Applied Statistics} 3 (2):
145--158.

\bibitem{} Greene, W. H. 1994. Accounting for excess zeros and sample
selection in Poisson and negative binomial regression models.\ Working Paper
No. 94-10, Department of Economics, Stern School of Business, New York
University.

\bibitem{} Greene, W. H., and D. A. Hensher. 2010. \textit{Modeling ordered
choices: A primer}.\ Cambridge University Press.

\bibitem{} Harris, M. N., and X. Zhao. 2007. A zero-inflated ordered probit
model, with an application to modelling tobacco consumption.\ \textit{%
Journal of Econometrics} 141 (2): 1073--1099.

\bibitem{} Hartman, R. S., M. Doane, and C.-K. Woo. 1991. Consumer
rationality and the status quo.\ \textit{Quarterly Journal of Economics}
106: 141--162.

\bibitem{} Heckman, J. J. 1978. Dummy endogenous variables in a simultaneous
equation system. \textit{Econometrica} 46: 931--959.

\bibitem{} Hern\'{a}ndez, A., F. Drasgow and V. Gonz\'{a}les-Rom\'{a}. 2004.
Investigating the functioning of a middle category by means of a
mixed-measurement model. \textit{Journal of Applied Psychology} 89 (4):
687-699.

\bibitem{} Kahneman, D., J. L. Knetsch, and R. H. Thaler. 1991. Anomalies:
the endowment effect, loss aversion, and status quo bias.\ \textit{Journal
of Economic Perspectives} 5 (1): 193--206.

\bibitem{} Kelley, M. E., and S. J. Anderson. 2008. Zero inflation in
ordinal data: incorporating susceptibility to response through the use of a
mixture model.\ \textit{Statistics in Medicine} 27: 3674--3688.

\bibitem{} Kulas, J. T. and A. A. Stachowski. 2009. Middle category
endorsement in odd-numbered Likert response scales: Associated item
characteristics, cognitive demands, and preferred meanings. \textit{Journal
of Research in Personality }43: 489-493.

\bibitem{} Lambert, D. 1992. Zero-inflated Poisson regression with an
application to defects in manufacturing.\ \textit{Technometrics} 34 (1):
1--14.

\bibitem{} MacKinnon, J. G. 1996. Numerical distribution functions for unit
root and cointegration tests.\ \textit{Journal of Applied Econometrics} 11:
601--618.

\bibitem{} Samuelson, W., and R. Zeckhauser. 1988. Status quo bias in
decision making.\ \textit{Journal of Risk and Uncertainty} 1: 7--59.

\bibitem{} Sirchenko, A. 2013. A model for ordinal responses with an
application to policy interest rate. National Bank of Poland Working Paper
No. 148.

\bibitem{} Small, K. 1987. A discrete choice model for ordered
alternatives.\ \textit{Econometrica} 55: 409--424.

\bibitem{} Vovsha, P. 1997. Application of cross-nested logit model to mode
choice in Tel Aviv, Israel, Metropolitan Area.\ \textit{Transportation
Research Record} 1607: 6--15.

\bibitem{} Vuong, Q. 1989. Likelihood ratio tests for model selection and
non-nested hypotheses.\ \textit{Econometrica} 57 (2): 307--333.

\bibitem{} Wen, C.-H., and F. Koppelman. 2001. The generalized nested logit
model. \textit{Transportation Research B} 35: 627--641.

\bibitem{} Wilde, J. 2000. Identification of multiple equation probit models
with endogenous dummy regressors.\ \textit{Economics Letters} 69 (3):
309--312.

\bibitem{} Winkelmann, R. 2008. \textit{Econometric analysis of count data}%
.\ 5$^{\text{th}}$ edition, Springer.

\bibitem{} 
\end{thebibliography}

\end{document}
